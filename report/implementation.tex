\chapter{Implementation Overview}

In this chapter, I consider the following simple Carat program:

\begin{lstlisting}
def goodbye(x)
  puts "Goodbye " + x
end

goodbye("Cruel World")
\end{lstlisting}

A \code{goodbye} method is defined and then called, causing `Goodbye Cruel World'\footnote{I find `Hello World' rather boring and thought it would be more fun to pay homage to Pink Floyd instead} to be printed as the output. I will explain how the program is executed, focussing on the overall design of the interpreter (therefore avoiding some of the finer details). Certain specific details will be covered in greater depth in \autoref{sec:implementation_specifics}.

\section{Parsing}

The first step is to parse the source code and convert it to an AST. This is done by the \code{Carat::}\-\code{Language}\-\code{Parser} class, which is largely produced by the parser generator, Treetop\footnote{\url{http://treetop.rubyforge.org/}}.

I chose Treetop because its grammer syntax is simple and readable. It uses Parsing Expression Grammars\footnote{\url{http://pdos.csail.mit.edu/~baford/packrat/}} which avoid the need to write a separate tokeniser. Whilst Treetop is not very efficient, it allowed me to quickly and easily develop and modify my parser. Given speed was not a goal of my implementation, I considered this a sensible trade-off.

Here is an excerpt from the grammar, which defines the syntax for a while loop:

\begin{lstlisting}[language=treetop]
rule while_expression
  'while' space condition:expression space? terminator
  contents:expression_list
  'end' <WhileExpression>
end
\end{lstlisting}

Characters in quotes are matched literally. The other items are references to other rules. Names before a colon apply a label to that part of the node. The question mark in `\code{space?}' makes the rule optional.

The parser produces a parse tree which consists of \code{Treetop::Runtime::SyntaxNode} objects. The class to be used to represent a \code{while_expression} node is specified by the `\code{<WhileExpression>}' at the end of the rule. Each different type of node has a \code{to_ast} method, which is used to recursively convert the parse tree to an AST made up of \code{Carat::AST::Node} objects.

\code{WhileExpression} is coverted like so:

\begin{lstlisting}
class WhileExpression < Treetop::Runtime::SyntaxNode
  def to_ast
    Carat::AST::While.new(location, condition.to_ast, contents.to_ast)
  end
end
\end{lstlisting}

In this simple example a \code{WhileExpression} parse tree nodes maps directly to a \code{While} AST node. This is not always the case, however -- some parse tree nodes perform additional syntax checks, or transform the data in some way. (For example \code{2 + 4} and \code{foo.bar} will be different types of parse tree nodes, but will both become \code{MethodCall} nodes in the AST.)

\section{Abstract Syntax Tree}

An AST node has a number of children (other nodes) and properties (static data). It also has an \code{eval} method which is discussed later. For example, the node for a method call is defined as:

\begin{lstlisting}
class MethodCall < Node
  child :receiver
  property :name
  child :arguments
end
\end{lstlisting}

These attributes can be used to print an AST as text. The method call in the example program is printed as:

\begin{lstlisting}[language=]
MethodCall[:goodbye]:
  receiver:
    nil
  arguments:
    ArgumentList:
      ArgumentList::Item[:normal]:
        expression:
          String["Cruel World"]
\end{lstlisting}

This can be read as a call to the method `goodbye'; the object receiving the call is not explicitly given and the argument list contains one item, which is the literal string ``Cruel World".

\section{The Runtime System}

\code{Carat::Runtime} is the starting point for running a program. When initialised it:

\begin{enumerate}
  \item Creates an empty \defn{stack of stacks}. When a program is being executed, its state is stored on a stack. But a program can load additional files which need to execute with their own stack before returning to the previous file. For this to happen, a ``stack of stacks" is needed.
  \item Creates an empty hash for \defn{constants}. Constants are globally defined, so are not stored on a stack.
  \item Creates an empty list of \defn{loaded files} to prevent the same file being loaded more than once.
  \item Runs the \code{KernelLoader} which sets up the core classes in the source language (\code{Object}, \code{Class}, \code{Array}, \code{String}, etc.). As far as is possible, these classes are defined directly in the source language. To optimise this process, parsing is done ahead-of-time and the AST nodes are stored as binary data in files which are then loaded directly.
\end{enumerate}

Another approach might be to have separate `environment' objects, which would independently represent the execution context for each different stack. I tried this, but it turned out to be far more practical to have an explicit, stored representation of the current execution state. This allows all the objects which need to access the execution state to simply hold a reference to the runtime. Otherwise, the current `environment' would need to be passed around a lot, which would make the code significantly more verbose. The stack of stacks simply makes explicit what would be implicitly stored on the implementation language's stack anyway. This seems acceptable to me.

\section{The Stack}

Whenever the \defn{stack} is referred to, this is the stack at the top of the stack of stacks. The stack contains one or more \defn{frames}. Frames can contain:

\begin{enumerate}
  \item A \defn{scope}, which stores the values of local variables
  \item A \defn{call}, which represents a method or lambda call
  \item A \defn{failure continuation}, which describes what to do when an exception is raised (this is not discussed until \autoref{sec:exceptions})
\end{enumerate}

All of these values are optional because they can all change independently of each other during program execution. But in practice a frame should have at least one or it will be quite useless.

The ``current scope", ``current call" or ``current failure continuation" refers to whichever value is in a frame closest to the top of the stack.

\section{The Object Model}
\label{sec:object_model}

Carat follows Ruby's object model closely. Everything is an object, and all objects are represented by the class \code{Carat::Data::ObjectInstance} in the implementation language. There are various other classes which implement behaviour for specific types of object, but all are subclasses of \code{ObjectInstance}. \autoref{fig:data_object_hierarchy} shows the inheritance hierarchy.

To understand the diagram, consider the class \code{Module} in the source language, which is represented by a \code{ModuleClass} instance in the implementation language. \code{ModuleClass} implements some specific behaviour which relates to the \code{Module} class. In the source language, \code{Module} is a subclass of \code{Object}, so in the implementation language, \code{ModuleClass} is a subclass of \code{ObjectClass}. \code{Object} is a type of class, so \code{ObjectClass} is a subclass of \code{ClassInstance}. All classes are a type of module, and all modules are a type of object, so \code{ClassInstance} subclasses \code{ModuleInstance} which then finally subclasses \code{ObjectInstance}.

This can be confusing, but the basic principle is simple: the \code{Module} class should have any behaviour which is given to the \code{Object} class, or any general class, or any general module, or any general object.

\begin{figure}
\begin{center}
\begin{tikzpicture}
[
  level distance=10mm,
  every node/.style={class,anchor=north},
  level 1/.style={sibling distance=40mm},
  level 3/.style={sibling distance=30mm}
]
\node {ObjectInstance}
  child { node {ModuleInstance}
    child { node {ClassInstance}
      child { node {ObjectClass}
        child { node {ModuleClass}
          child { node {ClassClass}
            child { node {SingletonClassClass} }
          }
        }
      }
      child { node [multiline node] {ArrayClass, \\ StringClass, \\ FixnumClass, \\ ...} }
      child { node [multiline node] {IncludeClass- \\ Instance} }
      child { node [multiline node] {SingletonClass- \\ Instance} }
    }
  }
  child { node [multiline node] {ArrayInstance, \\ StringInstance, \\ FixnumInstance, \\ ...} };
\end{tikzpicture}
\caption{Implementation language inheritance hierarchy of \movingcode{Carat::Data} classes}
\label{fig:data_object_hierarchy}
\end{center}
\end{figure}

\subsection{\movingcode{ObjectInstance}}

\code{ObjectInstance} is the implementation language class which represents a source language \textit{object}. Instances are created with the following signature:

\begin{lstlisting}
ObjectInstance.new(runtime, klass)
\end{lstlisting}

All \code{Carat::Data} objects hold a reference to the runtime they exist in. This allows them to perform operations which depend on the current execution state. The \code{klass} parameter is for an object representing the class of the instance (this spelling is used to avoid conflicts with the Ruby keyword \code{class}). When initialised, an \code{ObjectInstance} is assigned a unique numeric identifier.

Each object has a table of \defn{instance variables} mapping names to values.

Objects can have \defn{singleton methods}. These methods are specific to individual instances, so if there are two objects of the same class, \code{a} and \code{b}, and \code{a} defines a singleton method, \code{b} will not have that method.

Objects don't store their own singleton methods. Instead, these are stored in the method table of a \defn{singleton class}. This special class doesn't exist by default, but the first time a singleton method is defined, it will be created. The object's original class (also called the `real class') then becomes the superclass of its singleton class (\autoref{fig:singleton_class_creation}). This ensures the object can still access methods defined by its original class.

\begin{figure}
\begin{center}
\begin{tikzpicture}[node distance=15mm,
                    every edge/.append style={->},
                    every node/.style={minimum width=10mm}]

\begin{scope}[yshift=1cm]
\node[object]                (obj)   {apple};
\node[class,right=of obj]    (class) {Apple}
  edge [<-] node[auto,swap] {klass} (obj);
\node[class,above=of class]  (dots)  {...}
  edge [<-] node[auto,swap] {super} (class);
\end{scope}

\begin{scope}[xshift=8cm]
\node[object]                 (obj')    {apple};
\node[sclass,right=of obj']    (sclass') {apple'}
  edge [<-] node[auto,swap] {klass} (obj');
\node[class,above=of sclass'] (class')  {Apple}
  edge [<-] node[auto,swap] {super} (sclass');
\node[class,above=of class']  (dots')  {...}
  edge [<-] node[auto,swap] {super} (class');
\end{scope}

\begin{pgfonlayer}{background}
  \node (r1) [background,fit=(obj)(class)(dots)] {};
  \node (r2) [background,fit=(obj')(class')(sclass')(dots')] {};
\end{pgfonlayer}

\draw [diagram transition] (r1) -- (r2) node {Singleton class created};
\end{tikzpicture}
\caption{Singleton class creation. Instances are shown in blue, classes in grey, and singleton classes in red with a prime (') appended.}
\label{fig:singleton_class_creation}
\end{center}
\end{figure}

\subsection{\movingcode{ModuleInstance}}

\code{ModuleInstance} is the implementation language class which represents a source language \textit{module}. Instances are created with the following signature:

\begin{lstlisting}
ModuleInstance.new(runtime, klass, name = nil)
\end{lstlisting}

Modules are containers for methods. They cannot be instantiated, but can be included into other classes. They have a \defn{method table}, which maps method names to \code{MethodInstance} objects.

Modules are themselves objects, so they can have singleton methods, which are analogous to what would be called `static' or `class' methods in other languages. Unlike normal objects, modules create their singleton class immediately rather than waiting until it is needed; this prevents inconsistent or incorrect class pointers from occurring.

\subsection{\movingcode{ClassInstance}}

\code{ClassInstance} is the implementation language class which represents a source language \textit{class}. Instances are created with the following signature:

\begin{lstlisting}
ClassInstance.new(runtime, klass, superclass, name = nil)
\end{lstlisting}

Classes are like a modules, except that they \textit{can} be instantiated. They have a \defn{super} pointer, which is initially set to the value of the \code{superclass} argument. In general, however, the super pointer may be either a normal class or an `include class' (explained in \autoref{sec:module_inclusion}). The \code{superclass} \textit{method} returns the first normal class in the hierarchy of super pointers.

Just like modules, classes can have singleton (or `class') methods and the singleton class is created immediately. However, if a class \code{Square} is a subclass of \code{Shape}, we would expect \code{Square} to respond to any of the singleton methods which \code{Shape} responds to. The standard method for creating a singleton class would not allow this, so a slightly different strategy is used.

When a class creates its singleton class, it uses the singleton class of its superclass as the superclass of its singleton class. In this way, classes and their singleton classes are positioned in parallel (\autoref{fig:singleton_class_inheritance}) and inheritance of `class methods' works as expected.

In some contexts the word `metaclass' is used to refer to the class of a class. In Carat a `metaclass' would be the singleton class of a class, but I do not distinguish this: all objects \textit{can} have a singleton class, and all objects which are a module or a class \textit{do} in fact have a singleton class. A `class method' is nothing special; it is just a short way of saying `a singleton method of an object which is a class'.

\begin{figure}
\begin{center}
\begin{tikzpicture}[node distance=15mm,
                    every edge/.append style={->},
                    every node/.style={minimum width=10mm}]
\node[class] (square) {Square};
\node[sclass,right=of square] (square') {Square'}
  edge [<-] node[auto,swap] {klass} (square);

\node[class,above=of square] (shape) {Shape}
  edge [<-] node[auto] {super} (square);
\node[sclass,above=of square'] (shape') {Shape'}
  edge [<-] node[auto,swap] {super} (square')
  edge [<-] node[auto,swap] {klass} (shape);

\node[class,above=of shape] (dots) {...}
  edge [<-] node[auto] {super} (shape);
\node[sclass,above=of shape'] (dots') {...}
  edge [<-] node[auto,swap] {super} (shape');

\begin{pgfonlayer}{background}
  \node [background,fit=(square)(square')(dots)(dots')] {};
\end{pgfonlayer}

\end{tikzpicture}
\caption{Singleton classes reflecting the inheritance hierarchy. Classes are shown in grey, and singleton classes are shown in red with a prime (') appended.}
\label{fig:singleton_class_inheritance}
\end{center}
\end{figure}

\subsection{Four Core Classes}

The most important four classes in the language are \code{Object}, \code{Module}, \code{Class} and \code{SingletonClass}. Their relationships are somewhat complex, but can be summarised by some basic rules (which apply to all objects):

\begin{enumerate}
  \item For \textit{all classes except \code{Object}}, the superclass of the singleton class is the singleton class of the superclass (as explained above)
  \item \code{Object} does not have a superclass, but its singleton class must have one because otherwise the property that ``everything is an object" would be violated. Intuitively, the real class of any class must be \code{Class}, so this must be the superclass of \code{Object}'s singleton class.
  \item The class of \textit{any} singleton class is \code{SingletonClass}.
\end{enumerate}

The core classes and their relationships are constructed by \code{KernelLoader}. They are shown in \autoref{fig:core_relationships}.

This object model is quite similar to that of Smalltalk-80, which has certainly been an influence on Ruby. Some slightly terrifying diagrams of this can be found in Chapter 16 of \textit{Smalltalk-80: The Language and its Implementation}.

\begin{figure}
\begin{center}
\begin{tikzpicture}
[
  every path/.append style={->},
  class/.append style={minimum width=20mm},
  sclass/.append style={minimum width=20mm},
  point/.style={coordinate}
]

\matrix (matrix)
  [row sep=15mm]
  {
    &[5mm]
    \node (nil) {\textbf{nil}};
    &[20mm]
    &[20mm]
    &[5mm] \\
    
    &
    \node[class] (Object) {Object}; &
    \node[sclass] (Object') {Object'}; &
    \node[point] (p23) {}; &
    \node[point] (p24) {}; \\
    
    &
    \node[class] (Module) {Module}; &
    \node[sclass] (Module') {Module'}; &
    \node[point] (p33) {}; &
    & \\
    
    \node[point] (p40) {}; &
    \node[class] (Class) {Class}; &
    \node[sclass] (Class') {Class'}; &
    \node[point] (p43) {}; &
    & \\
    
    &
    \node[class,multiline node] (SingletonClass) {Singleton\\Class}; &
    \node[sclass,multiline node] (SingletonClass') {Singleton\\Class'}; &
    \node[point] (p53) {}; &
    & \\[-10mm]
    
    &
    \node[point] (p61) {}; &
    \node[point] (p62) {}; &
    \node[point] (p63) {}; &
    & \\[-10mm]
    
    \node[point] (p70) {}; &
    \node[point] (p71) {}; &
    &
    &
    \node[point] (p74) {}; & \\
  };

\draw (Object) -- node[right] {super} (nil);
\draw (Module) -- node[right] {super} (Object);
\draw (Class) -- node[right] {super} (Module);
\draw (SingletonClass) -- node[right] {super} (Class);

\draw (Object') -- ($(Object') + (0,10mm)$) -- node[above] {super} ($(p24) + (0,10mm)$) --
      (p74) -- (p70) -- ($(p40) - (0,10mm)$) -- ($(Class) - (5mm,10mm)$) -- ($(Class.south) - (5mm,0)$);

\draw (Module') -- node[right] {super} (Object');
\draw (Class') -- node[right] {super} (Module');
\draw (SingletonClass') -- node[right] {super} (Class');

\draw (Object) -- node[above] {klass} (Object');
\draw (Module) -- node[above] {klass} (Module');
\draw (Class) -- node[above] {klass} (Class');
\draw (SingletonClass) -- node[above] {klass} (SingletonClass');

\draw (Object') -- node[above] {klass} (p23) -- (p63) -- (p61) -- (SingletonClass);
\draw (Module') -- node[above] {klass} (p33);
\draw (Class') -- node[above] {klass} (p43);
\draw (SingletonClass') -- node[above] {klass} (p53);

\begin{pgfonlayer}{background}
  \node [background,fit=(matrix)] {};
\end{pgfonlayer}

\end{tikzpicture}
\caption{Class and superclass relationships between the four core classes and their singleton classes. Classes are shown in grey, and singleton classes are shown in red with a prime (') appended.}
\label{fig:core_relationships}
\end{center}
\end{figure}

\section{Continuation Passing Style}

The interpreter works out the result of a program by ``walking" the AST. An AST node with children probably needs to evaluate its child nodes before it can return its own answer. The obvious way to do this is to literally evaluate the children by calling the relevant evaluation methods, and then compute the answer to return.

This problem with this approach is that it does not allow for ``jumps". Jumps occur when the program needs to move to a different node in the AST without first returning the answer of the current node. This can happen when a return call or an exception is encountered. The problem is that the execution of the AST is intrinsically linked to the stack of the implementation language, which can't be directly manipulated.

To solve this, the interpreter is written in \defn{continuation passing style} (CPS). Any method involved in the evaluation of AST nodes expects to be called with a \defn{continuation}. Abstractly, this is an object which represents the computation still to be done once an answer has been found. In this implementation, continuations are represented as closures which expect one argument: the result of the node which calls the continuation.

Some AST nodes can return a result immediately, without further computation (such as the literal \code{Nil} node). In this case, the node will simply call the continuation, passing its immediate value as the argument.

Most nodes need to evaluate other nodes before they can produce an answer. In this case, instead of evaluating a given child node and waiting for the answer, they evaluate the child node and pass a continuation which captures what needs to be done with the answer. This way, nodes have a choice about whether to continue a computation by calling the continuation, or by calling some other continuation which jumps to another part of the program. The difference is illustrated in the following psuedocode:

\begin{lstlisting}[title={\textbf{Without CPS}}]
def eval
  child_value = eval_child(child_node)
  compute_value(child_value)
end
\end{lstlisting}

\begin{lstlisting}[title={\textbf{With CPS}}]
def eval(&continuation)
  eval_child(child_node) do |child_value|
    value = compute_value(child_value)
    continuation.call(value)
  end
end
\end{lstlisting}

In the first example, \code{eval_child} can't stop \code{compute_value} from being called when it returns.

In the second example, the \code{do .. end} block is a continuation which is passed to \code{eval_child}, containing the code to be executed once the answer of \code{eval_child} is available. Then \code{compute_value} is called with this child value, to produce the ultimate value of the \code{eval} method. The continuation of \code{eval} is finally called, passing the computed value as the answer. However, if \code{eval_child} wanted to prevent the program execution happening in this order, it could simply call a different continuation to the one it was given.

\section{Execution}

An AST is executed by passing its root node to \code{Runtime#execute}:

\begin{lstlisting}
def execute(root, scope = nil)
  with_stack { call_main_method(root, scope) }
end
\end{lstlisting}

Roughly speaking, the following happens:

\begin{enumerate}
  \item A new stack is pushed onto the stack of stacks
  \item The root node is wrapped in a special ``main" method, which is called
  \item The stack is removed from the stack of stacks
\end{enumerate}

\subsection{Trampoline}

\code{Runtime#with_stack} is defined as follows:

\begin{minipage}{\textwidth}
\begin{lstlisting}
def with_stack(&result)
  @stack_of_stacks << Stack.new
  while result.is_a?(Proc)
    result = result.call
  end
  @stack_of_stacks.pop
  result
end
\end{lstlisting}
\end{minipage}

One outcome of CPS is that all evaluation methods have calls to other evaluation methods in \defn{tail position}: the last line of the method. When a method call in tail position returns, no further computation is done before the method which made the call also returns. Language implementations with \defn{tail call optimisation} keep the stack size down by replacing the calling method on the stack with the method which is being called in tail position. Ruby does not implement tail call optimisation, so it is quite easy to write a Carat program which will exhaust the stack space. This is because in CPS, no evaluation method returns a result until the very last node is evaluated.

To solve this, \code{with_stack} uses a \defn{trampoline}. In certain places where an evaluation method would usually call another evaluation method, it instead wraps that call in a closure which is returned. This causes the stack in the implementation language to ``collapse" right back down to \code{with_stack}, which simply calls the closure to resume execution. The \code{while} loop does this repeatedly until an answer which is not a closure is returned.

Carat itself does not implement tail call optimisation.

\subsection{The main method}

The main method call is constructed like so:

\begin{minipage}{\textwidth}
\begin{lstlisting}
def call_main_method(contents, scope = nil)
  call  = MainMethodCall.new(contents.location)
  frame = Frame.new(scope || main_scope, call, default_failure_continuation)
  
  contents.runtime = self
  contents.eval_in_frame(frame, &identity_continuation)
end
\end{lstlisting}
\end{minipage}

If the \code{scope} argument is \code{nil}, then \code{scope || main_scope} causes the \code{main_scope} method to be called, which returns a default. Every scope must have a value for \code{self}, which is used when looking up instance variables and evaluating method calls with no explicit receiver. By default, \code{self} is an instance of \code{Object}.

A special \code{MainMethodCall} object is also created. This is just so the main method call is present on the stack and can be used when exceptions generate a backtrace.

An initial frame is created using the scope, call and default failure continuation. The AST node's \code{eval_in_frame} method is called, which adds the frame to the stack, evaluates the node, and then ensures the frame is subsequently removed from the stack. The continuation given does nothing with the final answer, because the end of the program has been reached.

\section{Sequential evaluation}

Returning to the example, the root node is an \code{ExpressionList} containing two items: a method definition and a method call. The items in an expression list are evaluated in turn, and the result of evaluating the last node becomes the result of the whole expression list.

This presents a challenge, because whilst the natural inclination would be to loop over the child nodes, CPS requires that evaluation calls only appear in tail position. To solve this, \code{Runtime} provides two high-level operations, \code{each} (for iteration) and \code{fold} (for accumulation). These translate a sequential operation on an array into a recursive one using continuations. For AST nodes, there is also \code{eval_fold} which accumulates the result of evaluating each node in an array.

\section{Method definition}

The first expression in the example is a \code{MethodDefinition}. Method definitions have a name (i.e. `\code{goodbye}') and potentially three child nodes: a \defn{receiver}, an \defn{argument pattern} and the \defn{contents}.

The receiver can be used to define a singleton method. For example, \code{a} is the receiver in `\code{def a.foo}', so the \code{foo} method will be defined in the singleton class of \code{a}. However, if no receiver is given explicitly the method is defined in the real class of the current \code{self}. In the example, this means that the \code{goodbye} method is added to the method table of the \code{Object} class.

The argument pattern specifies the format of the method's arguments, but this is discussed in detail in \autoref{sec:arguments}. The contents is an expression list containing the code `inside' the method.

A \code{MethodInstance} object is created from the name, argument pattern and contents, and then it is added to the method table of the relevant class.

\section{Method calls}

The second expression in the example is a \code{MethodCall}. Method calls have a name and two child nodes: the \defn{receiver} and the \defn{arguments}. Method calls are evaluated as follows:

\begin{enumerate}
  \item Evaluate the receiver. If there is no explicit receiver, the \code{self} object in the current scope is used.
  \item Ask the receiver object for the instance method with the given name. The advantage of the object model discussed in \autoref{sec:object_model} is that method look-up is very simple. When an object looks for an instance method, it asks its \code{klass} to look up that method. If a class contains the requested method in its method table, it returns it. Otherwise, it asks its \code{super}. Eventually, when there is no \code{super} (i.e. for \code{Object}), no method has been found so \code{nil} is returned.
  \item If the method is found, call the receiver object's \code{call} method which creates a \code{Call} object and \linebreak ``sends" it.
  \item If the method is not found, raise an exception (see \autoref{sec:exceptions})
\end{enumerate}

In the example there is no explicit receiver, so \code{self} is used, which is an \code{Object} instance. The instance asks its class (which is \code{Object}) to look up a method named `\code{goodbye}'. That method was added to \code{Object} in the previous expression, so this is successful and a \code{Call} is created.

The following attributes of a \code{Call} object are worth explaining:

\begin{enumerate}
  \item The \defn{callable} is an object representing the `thing' being called (this will either be a \code{MethodInstance} or a \code{LambdaInstance}).
  \item The \defn{scope} is the scope which the callable should be evaluated within. For method calls, this starts as an empty scope where \code{self} is the receiver object.
  \item The \defn{caller scope} is the scope from which the call was made. (The current scope of the runtime when the \code{Call} was created.)
  \item The \defn{argument list} is the \code{ArgumentList} AST node representing the arguments passed to the call. (It can also be an array of objects, for convenience when making calls within the implementation language.)
  \item The \defn{continuation} describes the computation to be done once the call has been sent.
\end{enumerate}

The following happens when a call is sent:

\begin{enumerate}
  \item A frame containing the call and its scope is pushed onto the stack
  \item The argument list is evaluated in the context of the caller scope to produce an \code{Arguments} object
  \item The arguments are matched to the argument pattern of the callable and assigned (see \autoref{sec:arguments} for a detailed explanation)
  \item The callable's contents are evaluated
\end{enumerate}

This is one of the key places where a closure is used to return to the trampoline method.

\section{Module inclusion}
\label{sec:module_inclusion}

When the \code{goodbye} call is sent, the contents of the \code{goodbye} method are executed. This is an expression list with a single item: a method call to \code{puts}. The (implicit) receiver of the \code{goodbye} call was \code{self}, which was an \code{Object} instance. So the scope inside the method has that same \code{Object} instance as the \code{self}. The \code{puts} call also has no explicit receiver, so again the receiver will be \code{self}.

However, the \code{Object} class does not define a \code{puts} method -- instead, it \defn{includes} the \code{Kernel} module, which does.

As explained above, the only way methods are found is by walking up the `super' chain. So if a class \code{Duck} includes the module \code{Quackable}, then \code{Quackable} needs to be inserted into the super chain of \code{Duck}. Suppose the original super of \code{Duck} is \code{Object}. One approach would be to change the super of \code{Duck} to \code{Quackable}, and set the super of \code{Quackable} to \code{Object}.

However, now suppose a \code{Square} class, which is a subclass of \code{Shape}, includes \code{Quackable}. \code{Square} would want to set \code{Quackable}'s super to \code{Shape}, but \code{Duck} would want it to be \code{Object}.

To solve this problem, modules are not directly inserted into the super chain. Instead, an \defn{include class} is created and the actual module's super is unchanged. The method table of an include class is a direct pointer to the method table of the module it represents. Similarly, another include class becomes the super of the singleton class, linking to the module's singleton class. This is shown in \autoref{fig:module_inclusion}.

\begin{figure}
\begin{center}

\tikzset{
  module inclusion diagram/.style={
    node distance=15mm,
    every edge/.append style={->},
    every node/.style={minimum width=10mm},
    split node/.style={
      rectangle split,
      rectangle split parts=2,
      rectangle split part align={left, left}
    },
    class/.append style={split node},
    sclass/.append style={split node},
    module/.append style={split node},
    remember picture
  }
}

\begin{tikzpicture}[module inclusion diagram]

\node[module] (Quackable) {\textbf{Quackable} \nodepart{two} method\_table};
\node[sclass,right=of Quackable] (Quackable') {\textbf{Quackable'} \nodepart{two} method\_table};
\node[class,above=of Quackable] (Duck) {\textbf{Duck} \nodepart{two} method\_table};
\node[sclass,right=of Duck] (Duck') {\textbf{Duck'} \nodepart{two} method\_table};
\node[class,above=of Duck] (Object) {\textbf{Object} \nodepart{two} method\_table};
\node[sclass,right=of Object] (Object') {\textbf{Object'} \nodepart{two} method\_table};

\draw[->] (Duck) -- node[auto,swap] {super} (Object);
\draw[->] (Duck') -- node[auto,swap] {super} (Object');
\draw[->] (Duck) -- node[auto] {klass} (Duck');
\draw[->] (Object) -- node[auto] {klass} (Object');

\draw[->] (Quackable) -- node[auto] {klass} (Quackable');

\begin{pgfonlayer}{background}
  \node (r1) [background,fit=(Object)(Quackable')] {};
\end{pgfonlayer}

\end{tikzpicture}

\vspace{3cm}

\begin{tikzpicture}[module inclusion diagram]

\node[class] (Object2) {\textbf{Object} \nodepart{two} method\_table};
\node[sclass,right=30mm of Object2] (Object2') {\textbf{Object'} \nodepart{two} method\_table};
\node[class,below=30mm of Object2] (Include2) {\textbf{\textit{IncludeClass}} \nodepart{two} method\_table};
\node[class,below=30mm of Object2'] (Include2') {\textbf{\textit{IncludeClass}} \nodepart{two} method\_table};
\node[class,below=of Include2] (Duck2) {\textbf{Duck} \nodepart{two} method\_table};
\node[sclass,below=of Include2'] (Duck2') {\textbf{Duck'} \nodepart{two} method\_table};

\node[module,above right=5mm and 6mm of Include2] (Quackable2) {\textbf{Quackable} \nodepart{two} method\_table};
\node[sclass,right=30mm of Quackable2] (Quackable2') {\textbf{Quackable'} \nodepart{two} method\_table};

\draw[->] (Duck2) -- node[auto,swap] {super} (Include2);
\draw[->] (Duck2') -- node[auto,swap] {super} (Include2');
\draw[->] (Include2) -- node[auto,near end,swap] {super} (Object2);
\draw[->] (Include2') -- node[auto,near end,swap] {super} (Object2');

\draw[->] (Duck2) -- node[auto] {klass} (Duck2');
\draw[->] (Object2) -- node[auto] {klass} (Object2');

\draw[->] (Quackable2) -- node[near end,auto] {klass} (Quackable2');

\draw[->,dashed] (Include2.two east) -- ++(3mm,0) -- ($(Quackable2.two west) - (3mm,0)$) -- (Quackable2.two west);
\draw[->,dashed] (Include2'.two east) -- ++(3mm,0) -- ($(Quackable2'.two west) - (3mm,0)$) -- (Quackable2'.two west);

\begin{pgfonlayer}{background}
  \node (r2) [background,fit=(Object2)(Duck2)(Quackable2')] {};
\end{pgfonlayer}
\end{tikzpicture}

\begin{tikzpicture}[overlay,remember picture]
  \draw [diagram transition] (r1) -- (r2)
    node[near end,draw=black!15,solid,text width=2cm,fill=black!3,inner sep=2mm] {\code{Quackable} included in \code{Duck}};
\end{tikzpicture}

\caption{Module inclusion. Classes are shown in grey, modules in green, and singleton classes are shown in red with a prime (') appended.}
\label{fig:module_inclusion}
\end{center}
\end{figure}

\section{Primitives}

The \code{puts} call has one argument: \code{"Cruel " + x}. This is actually a call to the \code{+} method of the string \code{"Cruel "}, with one argument, which is the variable \code{x}. Carat supports various special `operator' methods (\code{+}, \code{-}, \code{==}, \code{!=}, etc.), which are recognised by the parser and converted to standard \code{MethodCall} AST nodes. Unary operators are also supported; for example \code{-5} calls the \code{--} method on the receiver \code{5}.

The source language definition of \code{String#+} is:

\begin{lstlisting}
def +(other)
  Primitive.plus(other.to_s)
end
\end{lstlisting}

This is a \defn{primitive method call}. Primitives are the eventual way that a program can actually `do' something; everything else is just memory access and program flow. The \code{+} method is called in the usual way, and then the \code{Primitive.plus} call is where the primitive operation actually happens.

The source language \code{Primitive} class is represented by \code{Carat::Data::PrimitiveClass} in the implementation language. The primitive syntax is parsed in exactly the same way as any other method call, but \code{Primitive}\-\code{Class} overrides two methods which were originally defined by \code{ObjectInstance}:

\begin{enumerate}
  \item \textbf{\code{lookup_instance_method}}: Instead of returning a \code{MethodInstance} found in the method table of the class, it prepends `\code{primitive_}' to the method name and finds an implementation language method in the current \code{self} object with that name
  
  \item \textbf{\code{create_call}}: Instead of returning a \code{Call}, it returns a \code{PrimitiveCall}
\end{enumerate}

\code{PrimitiveCall#send} then evaluates the arguments in the standard way, and calls the implementation language method which provides the primitive behaviour.

Inside the \code{+} method, \code{self} is the \code{"Goodbye "} string, which is a \code{StringInstance} in the implementation language. So the \code{StringInstance#primitive_plus} method is looked up and called with the evaluated argument \code{other.to_s}. The primitive is defined as follows:

\begin{minipage}{\textwidth}
\begin{lstlisting}
def primitive_plus(other)
  yield real_klass.new(contents + other.contents)
end
\end{lstlisting}
\end{minipage}

The real (non-singleton) class of the string is found, and then a new instance is created, representing the concatenation. This new string is then passed to \code{yield}, which calls the continuation. (Note that \code{real_klass} will return an \textit{instance} of \code{StringClass}, so in this context \code{new} is an \textit{instance} method which creates a new instance of \code{StringInstance}.)

Once ``Goodbye " and ``Cruel World" have been added, the resulting string ``Goodbye Cruel World" becomes the argument to a \code{puts} call. This \code{puts} method also uses a primitive, but because it is defined in the included module \code{Kernel}, there is not actually a \code{primitive_puts} method in the implementation language \code{ObjectInstance} class. Instead, \code{primitive_puts} is defined in the \code{Carat::Data::KernelModule} module. When a module inclusion happens in the source language, Carat checks whether there is a corresponding module in the implementation language. If so, that module is included in the implementation language. So when \code{Object} includes \code{Kernel} in the source language, \code{ObjectInstance} includes \code{KernelModule} in the implementation language.

When \code{KernelModule#primitive_puts} is finally called, it outputs its argument, so ``Goodbye Cruel World" is printed out.
