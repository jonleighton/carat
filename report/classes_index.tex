\section{Index of Implementation Classes}

\subsection{Abstract syntax tree}

These classes are all defined in the \code{Carat::AST} namespace. In addition to those below, there is a \code{Printer} class which prints out an AST.

\subsubsection{Abstract Superclasses}

\begin{description}
	\item[\code{Node}] The superclass of all nodes
	
	\item[\code{ValueNode}] Nodes which evaluate to a single value (e.g. \code{Nil}, \code{False})
	
	\item[\code{MultipleValueNode}] Nodes which have some immediate (but not constant) value (e.g. \code{String}, \code{Fixnum})
	
	\item[\code{NamedNode}] Nodes which are defined by their name (e.g. \code{LocalVariable}, \code{InstanceVariable})
	
	\item[\code{NodeList}] Nodes containing a list of child nodes
	
	\item[\code{BinaryNode}] Nodes with two child nodes (e.g. \code{And}, \code{Or})
\end{description}

\subsubsection{Concrete nodes}

Node properties are shown in brackets.

\begin{description}
  \item[\code{If[condition, true_node, false_node]}] A conditional expression
  
  \item[\code{While[condition, contents]}] A while loop
  
  \item[\code{Begin[contents, rescue]}] A \code{begin} \code{...} \code{end} expression (with an optional \code{rescue} block)
  
  \item[\code{Rescue[error_type, exception_variable, contents]}] A \code{rescue} \code{...} \code{end} block
  
  \item[\code{And < BinaryNode}] An \code{&&} expression
  
  \item[\code{Or < BinaryNode}] An \code{||} expression
  
  \item[\code{True < ValueNode}] The \code{true} value
  
  \item[\code{False < ValueNode}] The \code{false} value
  
  \item[\code{Nil < ValueNode}] The \code{nil} value
  
  \item[\code{String < MultipleValueNode}] A literal string value
  
  \item[\code{Integer < MultipleValueNode}] A literal integer value
  
  \item[\code{Array < NodeList}] A literal array
  
  \item[\code{MethodCall[receiver, name, arguments]}] A method call
  
  \item[\code{ArgumentList < NodeList}] The argument list to a method call
  
  \item[\code{ArgumentList::Item[expression, type]}] An item in an \code{ArgumentList}
  
  \item[\code{Block[argument_pattern, contents]}] A literal block passed in the arguments of a method call
  
  \item[\code{ExpressionList < NodeList}] A list of nodes to be evaluated in order
  
  \item[\code{ModuleDefinition[name, contents]}] A module definition
  
  \item[\code{ClassDefinition[name, superclass, contents]}] A class definition
  
  \item[\code{MethodDefinition[receiver, name, argument_pattern, contents]}] A method definition
  
  \item[\code{ArgumentPattern < NodeList}] The argument pattern of a method definition
  
  \item[\code{ArgumentPattern::Item[assignee, type, default]}] An item in an \code{ArgumentPattern}
  
  \item[\code{Assignment}] An assignment of a value to a variable or assignment method
  
  \item[\code{LocalVariable < NamedNode}] A local variable
  
  \item[\code{LocalVariableOrMethodCall < NamedNode}] A local variable or method call, when it is ambiguous. A variable lookup is attempted first, falling back on a method call before failing
  
  \item[\code{InstanceVariable < NamedNode}] An instance variable
  
  \item[\code{Constant < NamedNode}] A constant
\end{description}

\subsection{Data objects}

These classes are all defined in the \code{Carat::Data} namespace. Indentation denotes the subclass hierarchy.

\begin{minipage}{\textwidth}
\begin{verbatim}
KernelModule
ObjectInstance
    ModuleInstance
        ClassInstance
            SingletonClassInstance
            IncludeClassInstance
            MethodClass
            LambdaClass
            PrimitiveClass
            FalseClassClass
            TrueClassClass
            NilClassClass
            ArrayClass
            StringClass
            ExceptionClass
        ObjectClass
            ModuleClass
                  ClassClass
                      SingletonClassClass
    MethodInstance
    LambdaInstance
    FalseClassInstance
    TrueClassInstance
    NilClassInstance
    ArrayInstance
    StringInstance
    ExceptionInstance
\end{verbatim}
\end{minipage}

\subsection{Parser}

\begin{description}
  \item[\code{Carat::SyntaxError}] Represents a general syntax error
  
  \item[\code{Carat::ParseError < SyntaxError}] A syntax error which occurs while trying to actual parse the source code (as opposed to when converting the parse tree to an AST)
  
  \item[\code{Carat::CommentParser}] A small parser which parses and removes any comments before the main parse is done
  
  \item[\code{Carat::LanguageParser}] The main parser
\end{description}

There are also lots of small classes representing parse tree nodes, but it is not really worth listing them all.

\subsection{Runtime}

These classes are all defined in the \code{Carat::Runtime} namespace.

\begin{description}
  \item[\code{Runtime}] Main runtime object
  
  \item[\code{KernelLoader}] Loads all the built-in objects for the runtime
  
  \item[\code{Scope}] Represents variable scopes
  
  \item[\code{Stack}] Represents the stack
  
  \item[\code{Frame}] Represents a stack frame
  
  \item[\code{Arguments}] Represents arguments to a call once they have been evaluated
  
  \item[\code{AbstractCall}] Common code for call objects
  
  \item[\code{PrimitiveCall < AbstractCall}] Represents a primitive call
  
  \item[\code{Call < AbstractCall}] Represents a method or lambda call
  
  \item[\code{MainMethodCall}] Represents the root `main' method call
\end{description}

\subsection{Other}

\begin{description}
  \item[\code{Carat::Location}] A source code location (file name, line number and column number)
  
  \item[\code{Carat::CaratError}] A general error within the interpreter
  
  \item[\code{Carat::REPL}] A simple read-eval-print-loop implementation, allowing direct interaction with the interpreter
\end{description}
