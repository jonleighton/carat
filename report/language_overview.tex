\section{Carat Language Overview}

\subsection{Data types}

\begin{description}
  \item[String:] \code{"double quoted"} or \code{'single quoted'}
  \item[Fixnum:] \code{6}, \code{-6} or \code{+6}
  \item[Array:] \code{[1, 2, 3]}
  \item[Boolean:] \code{true} and \code{false}
  \item[Nil:] \code{nil}
\end{description}

\subsection{Method names}

\begin{itemize}
  \item In general, method names have the pattern \code{[a-zA-Z_] [a-zA-Z0-9_]*}
  \item They may have \code{?}, \code{!} or \code{=} at the end
  \item There are some special `operator' method names: \code{===}, \code{<=>}, \code{[]=}, \code{==}, \code{!=}, \code{<=}, \code{>=}, \code{<<}, \code{>>}, \code{--} (unary minus), \code{++} (unary plus), \code{!!} (unary negation), \code{[]}, \code{<}, \code{>}, \code{+}, \code{-}, \code{*}, \code{/}
\end{itemize}

\subsection{Other names}

\begin{itemize}
  \item Class names (constants) begin with a capital letter
  \item Instance variables begin with a \code{@}
  \item Local variables begin with a lowercase letter
\end{itemize}

\subsection{Module definition}

\begin{lstlisting}
module ModuleName
  ...
end
\end{lstlisting}

\subsection{Class definition}

\begin{lstlisting}
class ClassName [< SuperClassName]
  ...
end
\end{lstlisting}

\subsection{Method definition}

\begin{lstlisting}
def method_name(argument_pattern)
  ...
end
\end{lstlisting}

\subsection{Blocks}

\begin{lstlisting}
foo { |argument_pattern| ... }

# or

foo do |argument_pattern|
  ...
end
\end{lstlisting}

\subsection{Control structures}

\begin{lstlisting}
if condition
  ...
elsif condition
  ...
else
  ...
end

while condition
  ...
end

begin
  ...
rescue [[ExceptionType] => [variable_name]]
  ...
end

first && second

first || second
\end{lstlisting}

\subsection{Comments}

\begin{lstlisting}
# Single line comment

##
Multi
line
comment
##
\end{lstlisting}

\subsection{Core Classes}

\subsubsection{\movingcode{Array}}

\begin{description}
  \item[\code{length}] Length of the array
  \item[\code{each(&block)}] Yields each item in the array to the block
  \item[\code{<<(value)}] Pushes a value onto the end of the array
  \item[\code{[i]}] Looks up the item \code{i}
  \item[\code{[i] = value}] Assigns a value at index \code{i}
  \item[\code{to_a}] Returns itself
  \item[\code{to_s}] Returns a string with each item on a separate line
  \item[\code{inspect}] Returns a string representation of the array contents
  \item[\code{map(&block)}] Returns a new array with each item being the result of yielding that item to the block
  \item[\code{join(joiner)}] Produces a string of each item joined with the \code{joiner} string
\end{description}

\subsubsection{\movingcode{Class}}

