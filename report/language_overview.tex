\section{Carat Language Overview}

\subsection{Data types}

\begin{description}
  \item[String:] \code{"double quoted"} or \code{'single quoted'}
  \item[Fixnum:] \code{6}, \code{-6} or \code{+6}
  \item[Array:] \code{[1, 2, 3]}
  \item[Boolean:] \code{true} and \code{false}
  \item[Nil:] \code{nil}
\end{description}

\subsection{Method names}

\begin{itemize}
  \item In general, method names have the pattern \code{[a-zA-Z_] [a-zA-Z0-9_]*}
  \item They may have \code{?}, \code{!} or \code{=} at the end
  \item There are some special `operator' method names: \code{===}, \code{<=>}, \code{[]=}, \code{==}, \code{!=}, \code{<=}, \code{>=}, \code{<<}, \code{>>}, \code{--} (unary minus), \code{++} (unary plus), \code{!!} (unary negation), \code{[]}, \code{<}, \code{>}, \code{+}, \code{-}, \code{*}, \code{/}
\end{itemize}

\subsection{Other names}

\begin{itemize}
  \item Class names (constants) begin with a capital letter
  \item Instance variables begin with a \code{@}
  \item Local variables begin with a lowercase letter
\end{itemize}

\subsection{Module definition}

\begin{lstlisting}
module ModuleName
  ...
end
\end{lstlisting}

\subsection{Class definition}

\begin{lstlisting}
class ClassName [< SuperClassName]
  ...
end
\end{lstlisting}

\subsection{Method definition}

\begin{lstlisting}
def method_name(argument_pattern)
  ...
end
\end{lstlisting}

\subsection{Blocks}

\begin{lstlisting}
foo { |argument_pattern| ... }

# or

foo do |argument_pattern|
  ...
end
\end{lstlisting}

\subsection{Control structures}

\begin{lstlisting}
if condition
  ...
elsif condition
  ...
else
  ...
end

while condition
  ...
end

begin
  ...
rescue [[ExceptionType] => [variable_name]]
  ...
end

first && second

first || second
\end{lstlisting}

\subsection{Comments}

\begin{lstlisting}
# Single line comment

##
Multi
line
comment
##
\end{lstlisting}

\subsection{Core Modules}

\subsubsection*{\movingcode{Comparable}}

\begin{tabular}{l p{10cm}}
  \textbf{Method Signature} & \textbf{Description} \\ \hline
  
  \code{<=>(other)} & Raises an exception; \code{<=>} must be implemented by the class including \code{Comparable}. \code{<=>} should compare self to \code{other}, returning \code{-1} when self is less than \code{other}, \code{0} when they are equal, and \code{1} when self is greater than \code{other}. \\
  \code{<(other)} & Returns \code{true} if self is less than \code{other}, \code{false} otherwise \\
  \code{>(other)} & Returns \code{true} if self is greater than \code{other}, \code{false} otherwise \\
  \code{<=(other)} & Returns \code{true} if self is less than or equal to \code{other}, \code{false} otherwise \\
  \code{>=(other)} & Returns \code{true} if self is greater than or equal to \code{other}, \code{false} otherwise \\
\end{tabular}

\subsubsection*{\movingcode{Kernel}}

\begin{tabular}{l p{10cm}}
  \textbf{Method Signature} & \textbf{Description} \\ \hline
  
  \code{raise(exception)} & Raises an exception \\
  \code{puts(obj = "\n")} & Converts \code{obj} to a string using \code{to_s} and outputs it \\
  \code{p(obj)} & Converts \code{obj} to a string using \code{inspect} and outputs it \\
  \code{lambda(&block)} & Alias for \code{Lambda.new} \\
  \code{yield(*args, &block)} & Calls the current block \\
  \code{return(value = nil)} & Returns from the current call, passing \code{value} as the result \\
  \code{require(file)} & Opens the \code{file}, executes its contents, and then returns to the execution of the current file \\
\end{tabular}

\subsection{Core Classes}

\subsubsection*{\movingcode{Array}}

\textbf{Superclass}: \code{Object}

\begin{tabular}{l p{10cm}}
  \textbf{Method Signature} & \textbf{Description} \\ \hline
  
  \code{length} & Length of the array \\
  \code{each(&block)} & Yields each item in the array to the block \\
  \code{<<(value)} & Pushes a value onto the end of the array \\
  \code{[i]} & Looks up the item \code{i} \\
  \code{[i] = value} & Assigns a value at index \code{i} \\
  \code{to_a} & Returns itself \\
  \code{to_s} & Returns a string with each item on a separate line \\
  \code{inspect} & Returns a string representation of the array contents \\
  \code{map(&block)} & Returns a new array with each item being the result of yielding that item to the block \\
  \code{join(joiner)} & Produces a string of each item joined with the \code{joiner} string \\
\end{tabular}

\subsubsection*{\movingcode{Class}}

\textbf{Superclass}: \code{Module}

\begin{tabular}{l l}
  \textbf{Method Signature} & \textbf{Description} \\ \hline
  
  \code{allocate} & Allocates space for a new instance and returns it \\
  \code{superclass} & Returns the superclass \\
  \code{include(module)} & Includes a module in the class \\
  \code{new(*args, &block)} & Creates a new instance, runs its initialiser and returns it \\
\end{tabular}

\subsubsection*{\movingcode{Exception}}

\textbf{Superclass}: \code{Object}

\begin{tabular}{l l}
  \textbf{Method Signature} & \textbf{Description} \\ \hline
  
  \code{initialize(message = "(no message)")} & Initialises an instance \\
  \code{to_s} & Returns the message string \\
  \code{backtrace} & Returns an array of strings representing the backtrace \\
\end{tabular}

\code{Exception} is subclassed by \code{StandardError}. \code{StandardError} is subclassed by \code{NameError}, \code{ArgumentError} and \code{RuntimeError}. \code{NameError} is subclassed by \code{NoMethodError}.

\subsubsection*{\movingcode{FalseClass}}

\textbf{Superclass}: \code{Object}

\begin{tabular}{l l}
  \textbf{Method Signature} & \textbf{Description} \\ \hline
  
  \code{to_s} & Returns the string ``false" \\
  \code{inspect} & Alias of \code{to_s} \\
\end{tabular}

\subsubsection*{\movingcode{Fixnum}}

\textbf{Superclass}: \code{Object} \\
\textbf{Includes}: \code{Comparable}

\begin{tabular}{l p{10cm}}
  \textbf{Method Signature} & \textbf{Description} \\ \hline
  
  \code{<=>(other)} & Compares self to \code{other}, return \code{-1}, \code{0}, \code{1} depending on whether \code{other} is greater than, equal to or less than self \\
  \code{+(other)} & Adds \code{other} to self \\
  \code{-(other)} & Subtracts \code{other} from self \\
  \code{--} & Returns the negative value of self (unary -) \\
  \code{++} & Returns the positive value of self (unary +) \\
  \code{to_s} & Prints out the number represented \\
  \code{inspect} & Alias of \code{to_s} \\
\end{tabular}

\subsubsection*{\movingcode{Lambda}}

\textbf{Superclass}: \code{Object}

\begin{tabular}{l p{10cm}}
  \textbf{Method Signature} & \textbf{Description} \\ \hline
  
  \code{call(*args, &block)} & Call the lambda \\
\end{tabular}

\subsubsection*{\movingcode{Module}}

\textbf{Superclass}: \code{Object}

\begin{tabular}{l p{10cm}}
  \textbf{Method Signature} & \textbf{Description} \\ \hline
  
  \code{name} & Returns the name of the module as a string \\
  \code{inspect} & Alias for \code{name} \\
  \code{to_s} & Alias for \code{to_s} \\
\end{tabular}

\subsubsection*{\movingcode{NilClass}}

\textbf{Superclass}: \code{Object}

\begin{tabular}{l p{10cm}}
  \textbf{Method Signature} & \textbf{Description} \\ \hline
  
  \code{inspect} & Returns ``nil" \\
  \code{to_s} & Returns empty string \\
\end{tabular}

\subsubsection*{\movingcode{Object}}

\textbf{Superclass}: \code{nil} \\
\textbf{Includes}: \code{Kernel}

\begin{tabular}{l p{10cm}}
  \textbf{Method Signature} & \textbf{Description} \\ \hline
  
  \code{initialize} & Does nothing (default behaviour for all objects) \\
  \code{==(other)} & Returns \code{true} if self is the same object as \code{other}, \code{false} otherwise \\
  \code{!=(other)} & Returns \code{true} if self is a different object to \code{other}, \code{false} otherwise \\
  \code{!!} & Returns \code{true} if self is \code{nil} or \code{false}, \code{false} otherwise (unary negation) \\
  \code{is_a?(test_class)} & Returns \code{true} if the class or a superclass of the class is the \code{test_class}, \code{false} otherwise \\
  \code{object_id} & Returns a unique identifier (\code{Fixnum}) for this object \\
  \code{class} & Returns the ``real" class of the object (i.e. \textit{not} a singleton class) \\
  \code{inspect} & Returns a string in the form ``<\textit{class\_name}:\textit{object\_id}>" \\
  \code{to_s} & Alias for \code{inspect} \\
\end{tabular}

\subsubsection*{\movingcode{String}}

\textbf{Superclass}: \code{Object}

\begin{tabular}{l p{10cm}}
  \textbf{Method Signature} & \textbf{Description} \\ \hline
  
  \code{+(other)} & Concatenates self with \code{other} \\
  \code{<<(other)} & Appends \code{other} to self \\
  \code{to_s} & Returns self \\
  \code{inspect} & Returns a string containing itself inside quotation marks \\
\end{tabular}

\subsubsection*{\movingcode{TrueClass}}

\textbf{Superclass}: \code{Object}

\begin{tabular}{l p{10cm}}
  \textbf{Method Signature} & \textbf{Description} \\ \hline
  
  \code{to_s} & Returns ``true" \\
  \code{inspect} & Alias for \code{to_s} \\
\end{tabular}
