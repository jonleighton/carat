\section{Introduction}

\subsection{Objectives}

My basic project idea was to produce some sort of programming language implementation. However, I didn't think it would be practical or useful to attempt to design a language from scratch; this can take years of careful thought in order to produce interesting results.

As I was already very familiar with Ruby\footnote{\url{http://www.ruby-lang.org/en/}}, which is a dynamic, high-level, object oriented programming language, I decided to implement a language similar to Ruby, using Ruby to write the implementation itself.

My language would seek to ask the question: \textit{what is the essence of Ruby?} I aimed to implement only the features which I considered essential to the nature of Ruby. In doing so, I would also carefully consider whether there were any changes or improvements which could be made to increase or refine its expressive power.

Many real-world language implementations are very complex due to heavy optimisation and the necessity of dealing with the myriad practical problems a production-quality programming language requires. They also tend to be written in lower level languages such as C or C++, for reasons of speed. Unfortunately this can make them difficult to study and understand. Without the burden of these requirements, my second objective was to produce a clean, well-written, concise implementation which neatly expressed the semantics of my language.

\subsection{Overview of Report}

It's fun to think up quirky names for projects, so I have chosen to name my language ``Carat" (the carat is a measure of purity in gold alloys).

This report starts by presenting a very basic example program and giving a detailed overview of how it is executed. This is used as a vehicle to explain how programs are parsed, how the runtime system is constructed, what object model is used, and how programs are executed. Method definitions and calls are described, as are `modules' and primitive calls.

Next, some features are discussed in more detail. It is shown how objects are created and how method arguments work. `Blocks' and lambda expressions are presented. The concluding part of this section explains how jumps in the control flow work, which are implemented as return calls and exceptions.

A short section then describes the testing strategy for the system.

The concluding section attempts to answer the original question about the `essence of Ruby', before highlighting areas where Carat differs from Ruby. Ideas for possible extensions to the project are given, before a final conclusion.

\subsection{Definitions}

\begin{description}
	\item[Implementation language] refers to the environment in which the implementation is written: the collection of Ruby code which forms the project
	\item[Source language] refers to the environment of the ``Carat" language actually being implemented
	\item[AST] is an acronym for ``Abstract Syntax Tree"
\end{description}

\subsection{Ruby syntax}

\begin{itemize}
  \item \code{Foo#bar} refers to the instance method \code{bar} of the class \code{Foo}
  
  \item \code{Foo.baz} refers to the class method \code{baz} of the class \code{Foo}
  
  \item \code{foo} refers to a local variable named \code{foo} (or a method)
  
  \item \code{@foo} refers to an instance variable named \code{foo}
  
  \item Blocks are a kind of lambda expression which can be passed as an argument to a method call, using either of two syntaxes:
  \begin{lstlisting}
  foo do |arglist|
    ...
  end
  
  # or
  
  foo { |arglist| ... }
  \end{lstlisting}
  
  The former is generally used when there are multiple lines in the block, and the latter when there is only one.
  
  \item If \code{bar} is a variable containing a lambda object, \code{foo(&bar)} will pass \code{bar} as the block to \code{foo}. So the following are equivalent:
  \begin{lstlisting}
  bar = lambda { |arglist| ... }
  foo(&bar)
  
  # and
  
  foo { |arglist| ... }
  \end{lstlisting}
  
  \item If a method is defined as \code{def foo(..., &block)}, then when the method is called, the \code{block} variable will contain the lambda object for the block, if there is one.
  
  \item If a method call is given a block, then \code{yield(arglist)} inside the method will call it.
\end{itemize}
