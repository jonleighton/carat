\section{The Implementation}

I explain my implementation in two steps. First, I present a simple program and explain its execution in depth. Then, I give examples of other language features which are worth explaining. I will not exhaustively cover all language features.

\subsection{A simple program}

In this section, I consider the execution of the following program:

\begin{lstlisting}
puts "Goodbye Cruel World"
\end{lstlisting}

This example is deliberately minimal. In explaining it, I will focus on the overall design of the interpreter, rather than the implementation of specific language features.

\subsubsection{Code layout}

The interpreter code is organised into the following directory structure:

\begin{dirlist}
  \item \textbf{lib/}
    \begin{dirlist}
      \item \textbf{ast/} - AST nodes
      \item \textbf{data/} - Classes representing objects in the target language
      \item \textbf{kernel/} - Core classes in the target language
      \item \textbf{parser/} - The parser and associated code
      \item \textbf{runtime/} - Classes representing run time state and behaviour
      \item \textbf{carat.rb} - Responsible for loading the interpreter
    \end{dirlist}
\end{dirlist}

\subsubsection{Parsing}

The first step is to parse the source code and convert it into an AST. This is done by the \code{Carat::LanguageParser} class, which is largely produced by the parser generator, Treetop\footnote{\url{http://treetop.rubyforge.org/}}.

I chose Treetop because its grammar format is intuitive. It uses Parsing Expression Grammars which avoid the need to write a separate tokeniser. Here is an excerpt from the grammar (in \file{parser/language.treetop}), defining the syntax for a while loop:

\begin{lstlisting}[language=treetop]
rule while_expression
  'while' space condition:expression space? terminator
  contents:expression_list
  'end' <WhileExpression>
end
\end{lstlisting}

Characters in single quotes are matched literally. The other items are references to rules. Names before a colon apply a label. The question mark in ``\code{space?}" makes that rule optional.

The parser produces a parse tree. Each node has a \code{to\_ast} method, which recursively converts it to an AST. In the above example, \code{<WhileExpression>} specifies the name of a class for the node, which is defined in \file{parser/nodes.rb}:

\begin{lstlisting}
class WhileExpression < Treetop::Runtime::SyntaxNode
  def to_ast
    Carat::AST::While.new(location, condition.to_ast, contents.to_ast)
  end
end
\end{lstlisting}

This is a simple example: more complex nodes may not map directly to an AST, and may need to do additional syntax checking.

\subsubsection{Abstract Syntax Tree}

Each AST node has a number of children (other nodes) and properties (static data). It also has an \code{eval} method which is discussed later. For example, the node for a method call is defined as:

\begin{lstlisting}
class MethodCall < Node
  child :receiver
  property :name
  child :arguments
end
\end{lstlisting}

These attributes can be used to print an AST as text. The example program's AST is printed as:

\begin{verbatim}
ExpressionList:
  MethodCall[:puts]:
    receiver:
      nil
    arguments:
      ArgumentList:
        ArgumentList::Item[:normal]:
          expression:
            String["Goodbye Cruel World"]
\end{verbatim}

This can be read as an expression list with one item: a call to the method `puts'. The object receiving the call is not explicitly given. The argument list contains one item, which is the literal string ``Goodbye Cruel World".

\subsubsection{Setting up the Runtime}

\code{Carat::Runtime} is the fundamental class responsible for executing a program. After the code is parsed, \code{setup\_environment} is called:

\begin{lstlisting}
def setup_environment
  # Initialize stacks
  @call_stack                 = []
  @scope_stack                = []
  @failure_continuation_stack = [default_failure_continuation]
  
  # Constants are defined globally
  @constants = {}
  
  # Load core classes
  KernelLoader.new(self).run
end
\end{lstlisting}

Three stacks are used:

\begin{enumerate}
  \item The \textit{call stack} contains \code{Call} objects which represent a call to a method or lambda (see [section])
  \item The \textit{scope stack} contains \code{Scope} objects which represent variable scopes (see [section])
  \item The \textit{failure continuation stack} contains objects representing what to do if an exception occurs (see [section])
\end{enumerate}

Constants are globally defined, so are not stored in \code{Scope} objects. \code{KernelLoader} sets up the core classes in the target language (\code{Object}, \code{Class}, \code{Array}, \code{String}, ...). To optimise this, the parsing is done ahead-of-time and the AST nodes are stored as binary data in files. These files are then loaded directly.

\subsubsection{The Object Model}

Carat follows Ruby's object model closely. Everything is an object, and all objects are represented by the class \code{Carat::Data::ObjectInstance} in the implementation language. There are various other classes which implement behaviour for certain types of objects, but all are subclasses of \code{ObjectInstance}. Through this, the inheritance hierarchy of data objects in the implementation language mirrors that of objects in the target language (figure \ref{fig:data_object_hierarchy}).

\begin{figure}
\begin{center}
\begin{tikzpicture}
[
  every node/.style={class},
  level 1/.style={sibling distance=40mm},
  level 3/.style={sibling distance=30mm},
  level 4/.style={level distance=10mm}
]
\node {ObjectInstance}
  child { node {ModuleInstance}
    child { node {ClassInstance}
      child { node {ObjectClass}
        child { node {ModuleClass}
          child { node {ClassClass}
            child { node {SingletonClassClass} }
          }
        }
      }
      child { node [multiline node] {ArrayClass, \\ StringClass, \\ FixnumClass, \\ ...} }
      child { node [multiline node] {IncludeClass- \\ Instance} }
      child { node [multiline node] {SingletonClass- \\ Instance} }
    }
  }
  child { node [multiline node] {ArrayInstance, \\ StringInstance, \\ FixnumInstance, \\ ...} };
\end{tikzpicture}
\caption{Inheritance hierarchy of \code{Carat::Data} classes}
\label{fig:data_object_hierarchy}
\end{center}
\end{figure}

\paragraph{\code{ObjectInstance}} An \code{ObjectInstance} is initialized with the following signature:

\begin{lstlisting}
ObjectInstance.new(runtime, klass)
\end{lstlisting}

All \code{Carat::Data} objects hold a reference to the runtime they exist in. The \code{klass} parameter is for an object representing the class of the instance (this spelling is used to avoid conflicts with the Ruby method \code{class}). When initialized, an \code{ObjectInstance} is assigned a unique numeric identifier.

Objects can have \defn{singleton methods}. These methods are specific to individual instances, so if there are two objects of the same class, \code{a} and \code{b}, and \code{a} defines a singleton method, \code{b} will not have that method.

Objects don't store their own singleton methods. Instead, a \defn{singleton class} is created when first required, and the object's original class becomes the superclass of the singleton class (figure \ref{fig:singleton_class_creation}). This ensures the object can still access methods defined by its original class.

\begin{figure}
\begin{center}
\begin{tikzpicture}[node distance=15mm,every edge/.append style={->}]

\begin{scope}[yshift=1cm]
\node[object]                (obj)   {apple};
\node[class,right=of obj]    (class) {Apple}
  edge [<-] node[auto,swap] {klass} (obj);
\node[dots,above=of class]  (dots)  {...}
  edge [<-] node[auto,swap] {super} (class);
\end{scope}

\begin{scope}[xshift=8cm]
\node[object]                 (obj')    {apple};
\node[sclass,right=of obj']    (sclass') {apple'}
  edge [<-] node[auto,swap] {klass} (obj');
\node[class,above=of sclass'] (class')  {Apple}
  edge [<-] node[auto,swap] {super} (sclass');
\node[dots,above=of class']  (dots')  {...}
  edge [<-] node[auto,swap] {super} (class');
\end{scope}

\begin{pgfonlayer}{background}
  \node (r1) [background,fit=(obj)(class)(dots)] {};
  \node (r2) [background,fit=(obj')(class')(sclass')(dots')] {};
\end{pgfonlayer}

\draw [->,shorten >=1mm,shorten <=1mm,dashed]
  (r1) -- (r2)
  node [above,align=center,midway,text width=1.5cm,font=\normalfont] {Singleton class created};
\end{tikzpicture}
\caption{Singleton class creation}
\label{fig:singleton_class_creation}
\end{center}
\end{figure}

It is useful to be able to identify the class from which an object was created, but calling \code{klass} might return a singleton class instead. So objects have a \code{real\_klass} method which returns the first \code{klass} or superclass of \code{klass} which is not a singleton class.

\paragraph{\code{ModuleInstance}} A module is a container of methods. It cannot be instantiated, but it can be included within other classes or modules. A \code{ModuleInstance} is initialized with the following signature:

\begin{lstlisting}
ModuleInstance.new(runtime, klass, name = nil)
\end{lstlisting}

A \code{ModuleInstance} has a \defn{method table}, which maps method names to \code{Method} objects.

Being a type of object, a module can have singleton methods which are analogous to ``class methods" in other languages. Unlike normal objects, modules create their singleton class immediately rather than waiting for it to be needed.

A module may also have a \defn{super} pointer, which facilitates the inclusion of one module within another through an \code{IncludeClassInstance} (see [section]).

\paragraph{\code{ClassInstance}} A class is like a module, but can be instantiated. A \code{ClassInstance} is initialized with the following signature:

\begin{lstlisting}
ClassInstance.new(runtime, klass, superclass, name = nil)
\end{lstlisting}

The \code{super} pointer gets set to the \code{superclass} provided. Generally, the \code{super} may be an include class or a normal class; the \code{superclass} method returns the first object in the \code{super} hierarchy which is a normal class.

Singleton classes are created in a slightly different way than for normal objects, because the inheritance hierarchy needs to be respected for ``class methods" as well as instance methods. When a class creates its singleton class, it uses the singleton class of its superclass as the superclass of its singleton class. In this way, classes and their singleton classes are positioned in parallel (figure \ref{fig:singleton_class_inheritance}).

\begin{figure}
\begin{center}
\begin{tikzpicture}[node distance=15mm,every edge/.append style={->}]
\node[class] (square) {Square};
\node[sclass,right=of square] (square') {Square'}
  edge [<-] node[auto,swap] {klass} (square);

\node[class,above=of square] (shape) {Shape}
  edge [<-] node[auto] {super} (square);
\node[sclass,above=of square'] (shape') {Shape'}
  edge [<-] node[auto,swap] {super} (square')
  edge [<-] node[auto,swap] {klass} (shape);

\node[dots,above=of shape] (dots) {...}
  edge [<-] node[auto] {super} (shape);
\node[dots,above=of shape'] (dots') {...}
  edge [<-] node[auto,swap] {super} (shape');

\begin{pgfonlayer}{background}
  \node [background,fit=(square)(square')(dots)(dots')] {};
\end{pgfonlayer}

\end{tikzpicture}
\caption{Singleton classes reflecting inheritance hierarchy}
\label{fig:singleton_class_inheritance}
\end{center}
\end{figure}

\paragraph{Four Core Classes} The most important four classes in the language are \code{Object}, \code{Module}, \code{Class} and \code{SingletonClass}. Their relationships are somewhat complex, but can be summarised by some basic rules:

\begin{enumerate}
  \item For \textit{all classes except \code{Object}}, the superclass of the singleton class is the singleton class of the superclass (as explained above)
  \item \code{Object} does not have a superclass; the superclass of \code{Object}'s singleton class is \code{SingletonClass}
  \item The class of \textit{any} singleton class is the singleton class of \code{SingletonClass}
\end{enumerate}

The core classes and their relationships are constructed by \code{KernelLoader}. They are shown in figure \ref{fig:core_relationships}.

\begin{figure}
\begin{center}
\begin{tikzpicture}
[
  every path/.append style={->},
  class/.append style={minimum width=20mm},
  sclass/.append style={minimum width=20mm},
  point/.style={coordinate}
]

\matrix (matrix)
  [row sep=15mm]
  {
    \node (nil) {nil};
    &[20mm]
    &[20mm]
    &[5mm] \\
    
    \node[class] (Object) {Object}; &
    \node[sclass] (Object') {Object'}; &
    \node[point] (p23) {}; &
    \node[point] (p24) {}; \\
    
    \node[class] (Module) {Module}; &
    \node[sclass] (Module') {Module'}; &
    \node[point] (p33) {}; &
    & \\
    
    \node[class] (Class) {Class}; &
    \node[sclass] (Class') {Class'}; &
    \node[point] (p43) {}; &
    & \\
    
    \node[class,multiline node] (SingletonClass) {Singleton\\Class}; &
    \node[sclass,multiline node] (SingletonClass') {Singleton\\Class'}; &
    \node[point] (p53) {}; &
    & \\[-10mm]
    
    &
    \node[point] (p62) {}; &
    \node[point] (p63) {}; &
    & \\[-10mm]
    
    \node[point] (p71) {}; &
    &
    &
    \node[point] (p74) {}; & \\
  };

\draw (Object) -- node[right] {super} (nil);
\draw (Module) -- node[right] {super} (Object);
\draw (Class) -- node[right] {super} (Module);
\draw (SingletonClass) -- node[right] {super} (Class);

\draw (Object') -- ($(Object') + (0,10mm)$) -- node[above] {super} ($(p24) + (0,10mm)$) -- (p74) -- (p71) -- (SingletonClass);
\draw (Module') -- node[right] {super} (Object');
\draw (Class') -- node[right] {super} (Module');
\draw (SingletonClass') -- node[right] {super} (Class');

\draw (Object) -- node[above] {klass} (Object');
\draw (Module) -- node[above] {klass} (Module');
\draw (Class) -- node[above] {klass} (Class');
\draw (SingletonClass) -- node[above] {klass} (SingletonClass');

\draw (Object') -- node[above] {klass} (p23) -- (p63) -- (p62) -- (SingletonClass');
\draw (Module') -- node[above] {klass} (p33);
\draw (Class') -- node[above] {klass} (p43);
\draw (SingletonClass') -- node[above] {klass} (p53);

\begin{pgfonlayer}{background}
  \node [background,fit=(matrix)] {};
\end{pgfonlayer}

\end{tikzpicture}
\caption{Class and superclass relationships between the four core classes and their singleton classes}
\label{fig:core_relationships}
\end{center}
\end{figure}
