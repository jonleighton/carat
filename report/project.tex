\documentclass[10pt,a4paper]{article}

%\usepackage{euler}
\usepackage[cm-default]{fontspec}
\usepackage{xunicode}
\usepackage{xltxtra}
\usepackage{ifpdf}
\usepackage[utf8]{inputenc}
\usepackage[UKenglish]{babel}
\usepackage[usenames,dvipsnames]{xcolor}
\usepackage[obeyspaces,spaces]{url}
\usepackage[breaklinks=true,colorlinks=true,linkcolor=MidnightBlue,urlcolor=OliveGreen]{hyperref}
\usepackage{listings}
\usepackage[a4paper]{geometry}
\usepackage{textcomp}
\usepackage{setspace}
\usepackage{amsmath}

\usepackage{tikz}
\usetikzlibrary{trees,chains,backgrounds,fit,calc,shapes.multipart}

\tikzset{
  generic node/.style={
    rectangle,rounded corners=2mm,
    very thick,
    text height=1.5ex,text depth=.25ex,
    align=center,
    text=black
  },
  grey node/.style={
    generic node,
    draw=black!50,
    top color=white,bottom color=black!20
  },
  red node/.style={
    generic node,
    draw=red!50!black!50,
    top color=white,bottom color=red!50!black!20
  },
  blue node/.style={
    generic node,
    draw=MidnightBlue!50!black!50,
    top color=white,bottom color=MidnightBlue!50!black!20
  },
  green node/.style={
    generic node,
    draw=OliveGreen!50!black!50,
    top color=white,bottom color=OliveGreen!50!black!20
  },
  object/.style={blue node,font=\ttfamily},
  class/.style={grey node,font=\ttfamily},
  sclass/.style={red node,font=\ttfamily},
  module/.style={green node,font=\ttfamily},
  multiline node/.style={
    text height=,text depth=
  },
  line/.style={
    thick,
    draw=black!50,
    color=black!50,
    text=black!70
  },
  every edge/.style={line},
  every path/.style={line,rounded corners},
  >=stealth,
  background/.style={
    fill=black!5,rounded corners
  },
  diagram transition/.style={
    ->,shorten >=1mm,shorten <=1mm,
    dashed,
    every node/.style={
      above,align=center,
      midway,text width=1.5cm,
      font=\normalfont,
      text=black
    }
  }
}

\definecolor{verylightgrey}{gray}{0.95}
\definecolor{lightgrey}{gray}{0.85}

\lstset{
  language=Ruby,                % choose the language of the code
  showstringspaces=false,       % underline spaces within strings
  tabsize=2,	                  % sets default tabsize to 2 spaces
  rulesepcolor=\color{Gray},
  basicstyle={\ttfamily\small},
  upquote=true,
  keywordstyle=\color{Mahogany}\bfseries,
  commentstyle=\color{MidnightBlue}\em,
  stringstyle=\color{RedOrange},
  aboveskip=\bigskipamount,
  belowskip=0pt,
  backgroundcolor=\color{verylightgrey},
  frame=single,
  frameround=tttt,
  rulecolor=\color{lightgrey}
}
\lstdefinelanguage{treetop}{
  keywords={rule,end,module,grammar},
  string=[b]{'},
  morestring=[b]{"},
  comment=[l]{\#}
}

\setmainfont[Mapping=tex-text-with-ligs]{GaramondClassico}
\setmonofont[Scale=0.8]{Bitstream Vera Sans Mono}
\newfontface\GaramondClassicoSC{GaramondClassicoSC}

\onehalfspace
\pagestyle{headings}

\numberwithin{figure}{section}

\def\gettitle{Carat: An interpreter written in Ruby}
\def\getauthor{Jonathan Leighton}
\def\getdate{\today}

\title{\gettitle}
\author{\getauthor}
\date{\getdate}

\hypersetup{
  pdftitle=\gettitle,
  pdfsubject=\gettitle,
  pdfauthor=\getauthor,
  pdfcreationdate=\getdate
}

% Use the url package to define a command for typesetting inline code
\DeclareUrlCommand\code{\def\UrlFont{\ttfamily \color{Mahogany}}}

% \url commands don't work in "moving arguments" (captions, sections, etc), so this is a fallback
% command which doesn't allow handle line breaks (but that shouldn't be a problem in these cases)
\newcommand{\movingcode}[1]{\texttt{\textcolor{Mahogany}{#1}}}

\newcommand{\defn}[1]{\emph{{#1}}}

\newenvironment{dirlist}{
  \begin{itemize}
  \setlength{\itemsep}{0pt}
  \setlength{\parskip}{0pt}
  \setlength{\parsep}{0pt}
  \renewcommand{\labelitemi}{}
  \renewcommand{\labelitemii}{}
  \renewcommand{\labelitemiii}{}
  \renewcommand{\labelitemiv}{}
}{
  \end{itemize}
}

\begin{document}
  \begin{titlepage}
\begin{center}

  {\GaramondClassicoSC {\huge \textbf{\gettitle}}} \\
  \bigskip
  {\Large
    \textbf{\getauthor} \\
    St Anne's College \\
    Oxford}\\
  \bigskip
  {\large \getdate}
  
	\input{wordcount.txt} words
	
	\vspace{1.5cm}
	
	{\Large \textbf{Abstract}}\\
  \parbox{10cm}{
  \bigskip
  In order to study the intricacies of programming language implementation, an interpreter for a 
  language similar to Ruby was written, dubbed ``Carat". The interpreter itself was written in
  Ruby, and a lot of effort went into keeping the level of complexity low, while still implementing
  a range of interesting and powerful language features.
  }
	
	\vspace{1cm}
	
	\textbf{Source code} \\
	\url{http://github.com/jonleighton/carat}

\end{center}
\end{titlepage}

	
	\tableofcontents
	
	\setlength{\parindent}{0pt}
  \setlength{\parskip}{2ex}
	
	\section{Introduction}

Objectives, overview of report

	\section{Implementation Overview}

In this section, I consider the execution of the following program:

\begin{lstlisting}
puts "Goodbye Cruel World"
\end{lstlisting}

This example is deliberately minimal. In explaining it, I will focus on the overall design of the interpreter, rather than the implementation of specific language features.

\subsection{Code layout}

The interpreter code is organised into the following directory structure:

\begin{dirlist}
  \item \textbf{lib/}
    \begin{dirlist}
      \item \textbf{ast/} - AST nodes
      \item \textbf{data/} - Classes representing objects in the target language
      \item \textbf{kernel/} - Core classes in the target language
      \item \textbf{parser/} - The parser and associated code
      \item \textbf{runtime/} - Classes representing run time state and behaviour
      \item \textbf{carat.rb} - Responsible for loading the interpreter
    \end{dirlist}
\end{dirlist}

\subsection{Parsing}

The first step is to parse the source code and convert it into an AST. This is done by the \code{Carat::LanguageParser} class, which is largely produced by the parser generator, Treetop\footnote{\url{http://treetop.rubyforge.org/}}.

I chose Treetop because its grammar format is intuitive. It uses Parsing Expression Grammars which avoid the need to write a separate tokeniser. Here is an excerpt from the grammar (in \file{parser/language.treetop}), defining the syntax for a while loop:

\begin{lstlisting}[language=treetop]
rule while_expression
  'while' space condition:expression space? terminator
  contents:expression_list
  'end' <WhileExpression>
end
\end{lstlisting}

Characters in single quotes are matched literally. The other items are references to rules. Names before a colon apply a label. The question mark in ``\code{space?}" makes that rule optional.

The parser produces a parse tree. Each node has a \code{to\_ast} method, which recursively converts it to an AST. In the above example, \code{<WhileExpression>} specifies the name of a class for the node, which is defined in \file{parser/nodes.rb}:

\begin{lstlisting}
class WhileExpression < Treetop::Runtime::SyntaxNode
  def to_ast
    Carat::AST::While.new(location, condition.to_ast, contents.to_ast)
  end
end
\end{lstlisting}

This is a simple example: more complex nodes may not map directly to an AST, and may need to do additional syntax checking.

\subsection{Abstract Syntax Tree}

Each AST node has a number of children (other nodes) and properties (static data). It also has an \code{eval} method which is discussed later. For example, the node for a method call is defined as:

\begin{lstlisting}
class MethodCall < Node
  child :receiver
  property :name
  child :arguments
end
\end{lstlisting}

These attributes can be used to print an AST as text. The example program's AST is printed as:

\begin{verbatim}
ExpressionList:
  MethodCall[:puts]:
    receiver:
      nil
    arguments:
      ArgumentList:
        ArgumentList::Item[:normal]:
          expression:
            String["Goodbye Cruel World"]
\end{verbatim}

This can be read as an expression list with one item: a call to the method `puts'. The object receiving the call is not explicitly given. The argument list contains one item, which is the literal string ``Goodbye Cruel World".

\subsection{Setting up the Runtime}

\code{Carat::Runtime} is the fundamental class responsible for executing a program. After the code is parsed, \code{setup\_environment} is called:

\begin{lstlisting}
def setup_environment
  # Initialize stacks
  @call_stack                 = []
  @scope_stack                = []
  @failure_continuation_stack = [default_failure_continuation]
  
  # Constants are defined globally
  @constants = {}
  
  # Load core classes
  KernelLoader.new(self).run
end
\end{lstlisting}

Three stacks are used:

\begin{enumerate}
  \item The \textit{call stack} contains \code{Call} objects which represent a call to a method or lambda (see [section])
  \item The \textit{scope stack} contains \code{Scope} objects which represent variable scopes (see [section])
  \item The \textit{failure continuation stack} contains objects representing what to do if an exception occurs (see [section])
\end{enumerate}

Constants are globally defined, so are not stored in \code{Scope} objects.

\code{KernelLoader} sets up the core classes in the target language (\code{Object}, \code{Class}, \code{Array}, \code{String}, ...). While some code is needed in the implementation language, most of this is done by interpreting class definitions written in the target language. To optimise this, the parsing is done ahead-of-time and the AST nodes are stored as binary data in files. These files are then loaded directly.

\subsection{The Object Model}
\label{sec:object_model}

Carat follows Ruby's object model closely. Everything is an object, and all objects are represented by the class \code{Carat::Data::ObjectInstance} in the implementation language. There are various other classes which implement behaviour for certain types of objects, but all are subclasses of \code{ObjectInstance}. Through this, the inheritance hierarchy of data objects in the implementation language mirrors that of objects in the target language (figure \ref{fig:data_object_hierarchy}).

\begin{figure}
\begin{center}
\begin{tikzpicture}
[
  every node/.style={class},
  level 1/.style={sibling distance=40mm},
  level 3/.style={sibling distance=30mm},
  level 4/.style={level distance=10mm}
]
\node {ObjectInstance}
  child { node {ModuleInstance}
    child { node {ClassInstance}
      child { node {ObjectClass}
        child { node {ModuleClass}
          child { node {ClassClass}
            child { node {SingletonClassClass} }
          }
        }
      }
      child { node [multiline node] {ArrayClass, \\ StringClass, \\ FixnumClass, \\ ...} }
      child { node [multiline node] {IncludeClass- \\ Instance} }
      child { node [multiline node] {SingletonClass- \\ Instance} }
    }
  }
  child { node [multiline node] {ArrayInstance, \\ StringInstance, \\ FixnumInstance, \\ ...} };
\end{tikzpicture}
\caption{Inheritance hierarchy of \code{Carat::Data} classes}
\label{fig:data_object_hierarchy}
\end{center}
\end{figure}

\subsubsection{\code{ObjectInstance}}

An \code{ObjectInstance} is initialized with the following signature:

\begin{lstlisting}
ObjectInstance.new(runtime, klass)
\end{lstlisting}

All \code{Carat::Data} objects hold a reference to the runtime they exist in. The \code{klass} parameter is for an object representing the class of the instance (this spelling is used to avoid conflicts with the Ruby method \code{class}). When initialized, an \code{ObjectInstance} is assigned a unique numeric identifier.

Each object has a hash of \defn{instance variables} mapping of their names to values.

Objects can have \defn{singleton methods}. These methods are specific to individual instances, so if there are two objects of the same class, \code{a} and \code{b}, and \code{a} defines a singleton method, \code{b} will not have that method.

Objects don't store their own singleton methods. Instead, a \defn{singleton class} is created when first required, and the object's original class becomes the superclass of the singleton class (figure \ref{fig:singleton_class_creation}). This ensures the object can still access methods defined by its original class.

\begin{figure}
\begin{center}
\begin{tikzpicture}[node distance=15mm,
                    every edge/.append style={->},
                    every node/.style={minimum width=10mm}]

\begin{scope}[yshift=1cm]
\node[object]                (obj)   {apple};
\node[class,right=of obj]    (class) {Apple}
  edge [<-] node[auto,swap] {klass} (obj);
\node[class,above=of class]  (dots)  {...}
  edge [<-] node[auto,swap] {super} (class);
\end{scope}

\begin{scope}[xshift=8cm]
\node[object]                 (obj')    {apple};
\node[sclass,right=of obj']    (sclass') {apple'}
  edge [<-] node[auto,swap] {klass} (obj');
\node[class,above=of sclass'] (class')  {Apple}
  edge [<-] node[auto,swap] {super} (sclass');
\node[class,above=of class']  (dots')  {...}
  edge [<-] node[auto,swap] {super} (class');
\end{scope}

\begin{pgfonlayer}{background}
  \node (r1) [background,fit=(obj)(class)(dots)] {};
  \node (r2) [background,fit=(obj')(class')(sclass')(dots')] {};
\end{pgfonlayer}

\draw [diagram transition] (r1) -- (r2) node {Singleton class created};
\end{tikzpicture}
\caption{Singleton class creation}
\label{fig:singleton_class_creation}
\end{center}
\end{figure}

It is useful to be able to identify the class from which an object was created, but calling \code{klass} might return a singleton class instead. So objects have a \code{real\_klass} method which returns the first \code{klass} or superclass of \code{klass} which is not a singleton class.

\subsubsection{\code{ModuleInstance}}

A module is a container of methods. It cannot be instantiated, but it can be included within other classes or modules. A \code{ModuleInstance} is initialized with the following signature:

\begin{lstlisting}
ModuleInstance.new(runtime, klass, name = nil)
\end{lstlisting}

A \code{ModuleInstance} has a \defn{method table}, which maps method names to \code{Method} objects.

Being a type of object, a module can have singleton methods which are analogous to ``class methods" in other languages. Unlike normal objects, modules create their singleton class immediately rather than waiting for it to be needed.

A module may also have a \defn{super} pointer, which facilitates the inclusion of one module within another through an \code{IncludeClassInstance} (see [section]).

\subsubsection{\code{ClassInstance}}

A class is like a module, but can be instantiated. A \code{ClassInstance} is initialized with the following signature:

\begin{lstlisting}
ClassInstance.new(runtime, klass, superclass, name = nil)
\end{lstlisting}

The \code{super} pointer gets set to the \code{superclass} provided. Generally, the \code{super} may be an include class or a normal class; the \code{superclass} method returns the first object in the \code{super} hierarchy which is a normal class.

Singleton classes are created in a slightly different way than for normal objects, because the inheritance hierarchy needs to be respected for ``class methods" as well as instance methods. When a class creates its singleton class, it uses the singleton class of its superclass as the superclass of its singleton class. In this way, classes and their singleton classes are positioned in parallel (figure \ref{fig:singleton_class_inheritance}).

\begin{figure}
\begin{center}
\begin{tikzpicture}[node distance=15mm,
                    every edge/.append style={->},
                    every node/.style={minimum width=10mm}]
\node[class] (square) {Square};
\node[sclass,right=of square] (square') {Square'}
  edge [<-] node[auto,swap] {klass} (square);

\node[class,above=of square] (shape) {Shape}
  edge [<-] node[auto] {super} (square);
\node[sclass,above=of square'] (shape') {Shape'}
  edge [<-] node[auto,swap] {super} (square')
  edge [<-] node[auto,swap] {klass} (shape);

\node[class,above=of shape] (dots) {...}
  edge [<-] node[auto] {super} (shape);
\node[sclass,above=of shape'] (dots') {...}
  edge [<-] node[auto,swap] {super} (shape');

\begin{pgfonlayer}{background}
  \node [background,fit=(square)(square')(dots)(dots')] {};
\end{pgfonlayer}

\end{tikzpicture}
\caption{Singleton classes reflecting inheritance hierarchy}
\label{fig:singleton_class_inheritance}
\end{center}
\end{figure}

\subsubsection{Four Core Classes}

The most important four classes in the language are \code{Object}, \code{Module}, \code{Class} and \code{SingletonClass}. Their relationships are somewhat complex, but can be summarised by some basic rules:

\begin{enumerate}
  \item For \textit{all classes except \code{Object}}, the superclass of the singleton class is the singleton class of the superclass (as explained above)
  \item \code{Object} does not have a superclass; the superclass of \code{Object}'s singleton class is \code{SingletonClass}
  \item The class of \textit{any} singleton class is the singleton class of \code{SingletonClass}
\end{enumerate}

The core classes and their relationships are constructed by \code{KernelLoader}. They are shown in figure \ref{fig:core_relationships}.

\begin{figure}
\begin{center}
\begin{tikzpicture}
[
  every path/.append style={->},
  class/.append style={minimum width=20mm},
  sclass/.append style={minimum width=20mm},
  point/.style={coordinate}
]

\matrix (matrix)
  [row sep=15mm]
  {
    \node (nil) {nil};
    &[20mm]
    &[20mm]
    &[5mm] \\
    
    \node[class] (Object) {Object}; &
    \node[sclass] (Object') {Object'}; &
    \node[point] (p23) {}; &
    \node[point] (p24) {}; \\
    
    \node[class] (Module) {Module}; &
    \node[sclass] (Module') {Module'}; &
    \node[point] (p33) {}; &
    & \\
    
    \node[class] (Class) {Class}; &
    \node[sclass] (Class') {Class'}; &
    \node[point] (p43) {}; &
    & \\
    
    \node[class,multiline node] (SingletonClass) {Singleton\\Class}; &
    \node[sclass,multiline node] (SingletonClass') {Singleton\\Class'}; &
    \node[point] (p53) {}; &
    & \\[-10mm]
    
    &
    \node[point] (p62) {}; &
    \node[point] (p63) {}; &
    & \\[-10mm]
    
    \node[point] (p71) {}; &
    &
    &
    \node[point] (p74) {}; & \\
  };

\draw (Object) -- node[right] {super} (nil);
\draw (Module) -- node[right] {super} (Object);
\draw (Class) -- node[right] {super} (Module);
\draw (SingletonClass) -- node[right] {super} (Class);

\draw (Object') -- ($(Object') + (0,10mm)$) -- node[above] {super} ($(p24) + (0,10mm)$) -- (p74) -- (p71) -- (SingletonClass);
\draw (Module') -- node[right] {super} (Object');
\draw (Class') -- node[right] {super} (Module');
\draw (SingletonClass') -- node[right] {super} (Class');

\draw (Object) -- node[above] {klass} (Object');
\draw (Module) -- node[above] {klass} (Module');
\draw (Class) -- node[above] {klass} (Class');
\draw (SingletonClass) -- node[above] {klass} (SingletonClass');

\draw (Object') -- node[above] {klass} (p23) -- (p63) -- (p62) -- (SingletonClass');
\draw (Module') -- node[above] {klass} (p33);
\draw (Class') -- node[above] {klass} (p43);
\draw (SingletonClass') -- node[above] {klass} (p53);

\begin{pgfonlayer}{background}
  \node [background,fit=(matrix)] {};
\end{pgfonlayer}

\end{tikzpicture}
\caption{Class and superclass relationships between the four core classes and their singleton classes}
\label{fig:core_relationships}
\end{center}
\end{figure}

\subsection{Continuation Passing Style}

The interpreter works out the result of a program by ``walking" the AST. An AST node with children probably needs to evaluate its child nodes before it can return its own answer. The obvious way to do this is to literally evaluate the children by calling the relevant evaluation methods, and then compute the answer to return.

This problem with this approach is it does not allow for ``jumps". Jumps occur when the program needs to move to a different node in the AST without first returning the answer of the current node. This can happen when a return call or an exception is encountered. The problem is that the execution of the AST is tied to the call stack of the implementation language, which can't be directly manipulated.

To solve this, the interepreter is written in \defn{continuation passing style} (CPS). Any method involved in the evaluation of AST nodes expects to be called with a \defn{continuation}. Abstractly, this is an object which represents the computation still to be done once an answer has been found. In this implementation, continuations are represented as lambdas, which expect one argument: the current result.

Some AST nodes can return a result immediately, without further computation (such as the literal \code{Nil} node). In this case, the node should simply call the continuation, passing its immediate value to the continuation.

Most nodes need to evaluate other nodes before they can produce an answer. In this case, instead of evaluating the child nodes and waiting for the answer, they evaluate the child node and pass a continuation which captures what needs to be done with the answer. This way, nodes have a choice about whether to continue a computation by calling the continuation, or by calling some other continuation which jumps to another part of the program.

\subsection{Execution}

An AST is executed by passing its root node to \code{Runtime\#execute}:

\begin{lstlisting}
def execute(root)
  current_result = call_main_method(root)
  
  while current_result.is_a?(Proc)
    current_result = current_result.call
  end
end
\end{lstlisting}

The root node is wrapped in a method named ``\code{main}" and then that method is called.

\subsubsection{Trampoline}

One outcome of CPS is that all evaluation methods have calls to other evaluation methods in \defn{tail position}: the last line of the method. When a method call in tail position returns, no further computation is done before the method which made the call also returns. Languages with \defn{tail call optimisation} keep the stack size down by replacing the calling method on the stack with the method which is being called in tail position. Ruby does not implement tail call optimisation, so it is quite easy to write a Carat program which will exhaust the stack space. This is because in CPS, no evaluation method returns a result until the very last node is evaluated.

To solve this, \code{execute} uses a \defn{trampoline}. In certain places where an evaluation method would usually call another evaluation method, it instead wraps that call in a lambda which is returned. This causes the stack in the implementation language to ``collapse" right back down to \code{execute}, which simply calls the lambda to resume execution. The \code{while} loop does this repeatedly until an answer which is not a lambda is returned.

\subsubsection{The main scope}

Every method must be executed within a \code{Scope}. Every \code{Scope} must have a value for \code{self}, which is used when looking up instance variables and evaluating method calls with no explicit receiver. For the main method, a new instance of \code{Object} is used as the \code{self} value of the execution scope.

\subsubsection{Calling the main method}

Methods and lambdas are called by \code{Runtime\#call}:

\begin{lstlisting}
def call(location, callable, scope, argument_list = [], &continuation)
  call = Call.new(self, location, callable, scope, argument_list)
  call.send(&continuation)
end
\end{lstlisting}

A \code{Call} object represents the call, and its \code{send} method actually evaluates that call. Roughly speaking, the following happens when a method call is sent:

\begin{enumerate}
  \item The argument list is evaluated to produce the argument values
  \item The argument values are assigned within the execution scope, according to the argument pattern of the callable (this is discussed in section ?)
  \item The \code{Call} is pushed onto the call stack
  \item The execution scope is pushed onto the scope stack
  \item The call's \defn{return continuation} is set, which pops the call and execution stacks before continuing with the result
  \item The \code{eval} method for the AST node of the method contents is called
\end{enumerate}

\code{Call\#send} is one of the places where a lambda is used to collapse the implementation language call stack.

\subsection{Sequential evaluation}

Returning to the example, the top node is an expression list. Expression list nodes contain an array of child nodes, which are evaluated in turn. The result of the last node is the result of the expression list.

This presents a challenge, because whilst the natural inclination would be to loop over the child nodes, CPS requires that evaluation calls only appear in tail position. To solve this, \code{Runtime} provides two high-level operations, \code{each} (for iteration) and \code{fold} (for accumulation). These translate a sequential operation on an array into a recursive one using continuations. For AST nodes, there is also \code{eval\_fold} which accumulates the result of evaluating each node in an array.

\subsection{Method calls}

The only item in the expression list is a \code{MethodCall[:puts]} with no receiver and and argument list containing ``Goodbye Cruel World". Method calls are evaluated as follows:

\begin{enumerate}
  \item Evaluate the receiver. If there is no explicit receiver, the \code{self} object in the current scope (the one at the top of the scope stack) is used.
  \item Check whether the receiver object has an instance method with the given name. The advantage of the object model discussed in section \ref{sec:object_model} is that method look-up is very simple. When an object looks for an instance method, it asks its \code{klass} to look up that method. If the class contains the method in its method table, it returns it. Otherwise, it asks its \code{super}. Eventually, when there is no \code{super} (i.e. for \code{Object}), no method has been found so \code{nil} is returned.
  \item If the method is found, use the receiver object's \code{call} method (which then uses \code{Runtime\#call}) to dispatch the call.
  \item Otherwise, an exception is raised (see section ?)
\end{enumerate}

\subsection{Module inclusion}

In the example, \code{self} is an \code{Object} instance. \code{Object} does not have a \code{puts} method, but it includes the \code{Kernel} module, which does.

As explained above, the only way methods are found is by walking up the \code{super} chain. So, when a module is included within another module or class, it must be inserted within the \code{super} chain. If inserted directly, the module's \code{super} would need to take the original value of the \code{super} of the object including the module. This would be problematic if the module was included more than once.

Instead, an \defn{include class} is created and the actual module's \code{super} is unchanged. The method table of an include class is a direct reference to the method table of the module it represents. Note that this implementation means including a module \textit{does not} make its singleton methods available. Module inclusion is shown in figure \ref{fig:module_inclusion}.

\begin{figure}
\begin{center}
\begin{tikzpicture}[
  node distance=10mm,
  every edge/.append style={->},
  every node/.style={minimum width=10mm},
  split node/.style={
    rectangle split,
    rectangle split parts=2,
    rectangle split part align={left, left}
  },
  class/.append style={split node},
  module/.append style={split node}
]

\begin{scope}[yshift=-1cm]
  \node[class] (Object) {\textbf{Object} \nodepart{two} method\_table};
  \node[class,below=of Object] (Duck) {\textbf{Duck} \nodepart{two} method\_table}
    edge [->] node[auto,swap] {super} (Object);

  \draw ($(Object) + 0.5*(Duck) - 0.5*(Object) + (3,0)$)
    node[module] (Quackable) {\textbf{Quackable} \nodepart{two} method\_table};
\end{scope}

\begin{scope}[xshift=8cm]
  \node[class] (Object') {\textbf{Object} \nodepart{two} method\_table};
  \node[class,below=of Object'] (Include') {\textbf{\textit{IncludeClass}} \nodepart{two} method\_table}
    edge [->] node[auto,swap] {super} (Object');
  \node[class,below=of Include'] (Duck') {\textbf{Duck} \nodepart{two} method\_table}
    edge [->] node[auto,swap] {super} (Include');
  
  \node[module,right=of Include'] (Quackable') {\textbf{Quackable} \nodepart{two} method\_table};
  
  \draw[->,dashed] (Include'.two east) -- (Quackable'.two west);
\end{scope}

\begin{pgfonlayer}{background}
  \node (r1) [background,fit=(Object)(Duck)(Quackable)] {};
  \node (r2) [background,fit=(Object')(Duck')(Quackable')] {};
\end{pgfonlayer}

\draw [diagram transition] (r1) -- (r2) node {\code{Quackable} included in \code{Duck}};

\end{tikzpicture}
\caption{Module inclusion}
\label{fig:module_inclusion}
\end{center}
\end{figure}

\subsection{Primitives}

The method definition for \code{Kernel\#puts} (in \file{kernel/kernel.carat}) is:

\begin{lstlisting}
def puts(obj = "\n")
  Primitive.puts(obj)
end
\end{lstlisting}

This is a \defn{primitive method call}. Primitives are the eventual way that a program can actually ``do" something; everything else is just memory access and method calls. In this case, the \code{puts} method needs to output some text, so it uses the \code{puts} primitive.

The \code{Primitive} class is represented by \code{Carat::Data::PrimitiveClass} in the implementation language. The primitive syntax is parsed in exactly the same way as any other method call, but \code{PrimitiveClass} overrides the default \code{ObjectInstance\#call} method to do the following:

\begin{enumerate}
  \item Evaluate the argument list
  \item Get the object representing the current \code{self}
  \item Prepend ``\code{primitive\_}" to the method name and call that method on the object, passing the evaluated arguments
\end{enumerate}

For the example, \code{self} is an \code{Object}, but the method is in \code{Kernel}. The \code{primitive\_puts} method is actually defined the \code{Carat::Data::KernelModule} module. When a module inclusion happens in the target language, the implementation checks whether there is a corresponding module in the implementation language. If there is, that module is included in the implementation language. For example, when \code{Object} included \code{Kernel} in the target language, \code{ObjectInstance} includes \code{KernelModule} in the implementation language.

When \code{KernelModule\#primitive\_puts} is finally called, it converts the object given as an argument to a Ruby string, and uses Ruby's \code{Kernel.puts} method to output it.

	\section{Implementation Specifics}

\subsection{Object creation}

\subsection{Methods: argument patterns and lists}

\subsection{Blocks, lambdas and yield}

\subsection{Exceptions}

	\section{Testing}

Initially, I wrote simple test programs and used a testing framework in Ruby to assert that the output was as expected. This strategy was okay, but it did mean that the tests were quite verbose, and took a long time to run due to the overhead of loading the interpreter for every test.

Once the implementation had become mature enough, I was able to write a very simple testing framework actually in the Carat language. I named it ``CSpec", as it was inspired by a Ruby testing framework called RSpec\footnote{\url{http://rspec.info/}}.

I then systematically worked through all the different objects and methods in the implementation, writing tests for them. I also tested various syntax features. Below is part of the specification for \code{Fixnum}:

\begin{lstlisting}
describe "Fixnum" do
  it "should use the same object for two instances of the same number" do
    24.object_id.should == 24.object_id
  end
  
  it "should support the negative unary prefix" do
    -6.should == (0 - 6)
  end
  
  it "should support the positive unary prefix" do
    +3.should == 3
  end
  
  it "should add two numbers" do
    (4 + 2).should == 6
  end
  
  it "should subtract two numbers" do
    (7 - 2).should == 5
  end
  
  ...
end
\end{lstlisting}

\newpage
When the \code{Fixnum} specification is run with CSpec, it outputs:

\begin{minipage}{\textwidth}
\begin{lstlisting}[language=]
Fixnum
 - should use the same object for two instances of the same number
 - should support the negative unary prefix
 - should support the positive unary prefix
 - should add two numbers
 - should subtract two numbers
 - should multiply two numbers
 - should divide two numbers (with integer division)
 - should return its value as a string with to_s
 - should return its value as a string with inspect

Fixnum#<=>
 - should return -1 for 1 <=> 2
 - should return 1 for 2 <=> 1
 - should return 0 for 1 <=> 1

12 examples, 15 assertions
\end{lstlisting}
\end{minipage}

If I intentionally break one of the examples, an exception is raised:

\begin{lstlisting}[language=]
Fixnum
 - should use the same object for two instances of the same number
 - should support the negative unary prefix

FAILED: -6 (actual) did not match 6 (expected)
spec/cspec.carat at line 118, col 7 in <method:==>
spec/fixnum_spec.carat at line 9, col 5 in <lambda>
spec/cspec.carat at line 83, col 7 in <method:call>
spec/cspec.carat at line 83, col 7 in <method:run>
spec/cspec.carat at line 67, col 7 in <lambda>
spec/cspec.carat at line 62, col 5 in <method:call>
spec/cspec.carat at line 62, col 5 in <method:each>
spec/cspec.carat at line 62, col 5 in <method:run>
spec/cspec.carat at line 38, col 7 in <lambda>
spec/cspec.carat at line 37, col 5 in <method:call>
spec/cspec.carat at line 37, col 5 in <method:each>
spec/cspec.carat at line 37, col 5 in <method:run>
 at line 1, col 1 in main

...
\end{lstlisting}

\newpage
There are also tests for semantic features, such as the following for an \code{if} expression:

\begin{lstlisting}
describe "An if expression" do
  it "should run the first branch and not the second branch if the condition is true" do
    if true
      a = "PASS"
    else
      flunk
    end
    a.should == "PASS"
    
    if true
      b = "PASS"
    end
    b.should == "PASS"
  end
  
  ...
end
\end{lstlisting}

In total there are 131 examples, comprising 172 separate assertions. It would be bold to claim that this means absolutely all functionality is tested and working, but given the experimental nature of this project I think it demonstrates an acceptable level of stability. The full output can be found in \autoref{sec:test_output}.

	\section{Conclusions}

\subsection{Comparison to Ruby}
\label{sec:ruby_comparison}

Ruby has been around since 1995, and has been in widespread use for many years. It is clearly far more mature and feature-complete than Carat, and much more suited to ``real world" programming tasks. The current most popular release, 1.8, is considered slow compared to other similar languages. Given that Carat is written in Ruby, it is obviously several times slower still.

Another point worth noting is that Carat does very little in terms of error checking. Many methods will produce unexpected results if called with invalid arguments.

However, the aim of Carat was not to produce a replacement to Ruby, but to produce a minimal language similar to it. There are a number of areas in particular where Carat differs to Ruby, in my view in a positive way.

\subsubsection{Lambdas}

Ruby has two different kinds of `lambda'. They are both instances of the \code{Proc} class, but depending on how the object is created, passing the wrong number of arguments when calling may or may not cause an error.

Additionally, without going into too much detail, argument assignment for blocks works slightly differently to how it works for methods. (Although this behaviour is changed in Ruby 1.9.)

In Carat, the \textit{only} difference between lambdas and methods is scoping, and there is only one type of lambda. I think this consistency is valuable, and it certainly makes the implementation simpler as code can be shared.

\subsubsection{Argument patterns}

In Ruby, argument patterns can only contain local variables. This often leads to object initialisers which look like this:

\begin{lstlisting}
def initialize(a, b, c = nil)
  @a, @b, @c = a, b, c
end
\end{lstlisting}

Allowing instance variables and methods in the argument pattern eliminates the repetition, and leads to more succinct code, which I find quite pleasing:

\begin{lstlisting}
def initialize(@a, @b, @c = nil)
end
\end{lstlisting}

\subsubsection{Module inclusion}

In Ruby, including a module only makes its instance methods available, which often leads to confusion. I like Carat's approach of making both instance and singleton (class) methods available.

On the other hand, Ruby supports inclusion of modules inside other modules, which Carat does not. It may be that there are additional complexities in this which I have not considered.

\subsection{Possible project extensions}

\subsubsection{A virtual machine}

At the moment Carat is purely an interpreter. It would be interesting to turn it into a bytecode compiler and interpreter, exploring how that changes the architecture and what the implementation challenges are. There would certainly be efficiency gains, particularly as continuation passing style is quite a slow strategy.

\subsubsection{Unicode syntax}

Pretty much every programming language in use today restricts its syntax to the ASCII character set. It would be interesting to see what gains could be made by allowing a wider set of characters. This would make programs more concise and readable, but would require special keyboards in order to write code quickly.

But I think the gains could be worth it. For example, \code{lambda} could be replaced with $\color{Mahogany} \lambda$, \code{!} (negation) could be replaced with $\color{Mahogany} \neg$ and \code{=} (assignment) could be replaced with $\color{Mahogany} \leftarrow$ (leaving \code{=} free for equality testing).

\subsubsection{Lazy evaluation}

Most operators in Carat (and Ruby) are implemented as method calls. However, \code{&&} and \code{||} in particular cannot be implemented as method calls because they need to `short circuit' depending on the value of the expression on the left of the operator. This could not be achieved through a method call as arguments are always evaluated before the call is made.

I don't think it would be a good idea to make the entire language use lazy evaluation, but it might be possible to implement a feature where argument patterns can use some sort of syntactic notation to specify that an argument should not be evaluated immediately. This would allow \code{&&} and \code{||} to be defined as methods. However, it may be that such a feature would easily lead to confusion, so it might not be a good idea.

\subsection{Overall Conclusions}

On the whole, I think this project has been successful in reaching its objectives. It has taught me a lot about programming languages. The implementation is concise and easy to follow; it weighs in at only just over 3000 lines of code. Writing this report has been useful because it forced me to think explicitly about design decisions, which has lead to several good refactorings.

It's a shame that it's unrealistic to produce serious language implementations in a language as expressive and powerful as Ruby, although Rubinius\footnote{\url{http://rubini.us/}} (a C++/Ruby implementation of Ruby) comes close.

	\section{References}

I have drawn inspiration from the following sources:

\begin{enumerate}
  \item Mike Spivey, my project supervisor.
  
  \item Patrick Farley's explanations of the Ruby object model at \url{http://www.klankboomklang.com/}. The Carat object model is based on Ruby's, although there are some differences.
  
  \item Smalltalk-80: The Language and its Implementation, Goldberg and Robson, Addison-Wesley, ISBN: 0-201-11371-6. This book describes a bytecode interpreter, but my implementation of primitives was inspired by it.
  
  \item Ruby, obviously.
\end{enumerate}

	
	\appendix
	\section{Carat Language Overview}



	\lstset{
  language=Ruby,
  tabsize=2,
  rulesepcolor=\color{Gray},
  basicstyle={\ttfamily\scriptsize},
  upquote=true,
  keywordstyle=\color{Mahogany}\bfseries,
  commentstyle=\color{MidnightBlue}\em,
  stringstyle=\color{RedOrange},
  aboveskip=\bigskipamount,
  belowskip=0pt,
  backgroundcolor=,
  frame=,
  frameround=,
  rulecolor=,
  breaklines=true
}

\newgeometry{
  hmargin=10mm,
  vmargin=20mm,
  twocolumn=true
}

\section{Code Listing}
\begin{lstlisting}[title={\small\ttfamily\bfseries ast/ast.rb},language=Ruby]
module Carat
  module AST
    require AST_PATH + "/printer"
    
    # ***** ABSTRACT SUPERCLASSES ****** #
    
    # The superclass of all AST nodes
    class Node
      class << self
        def attributes
          @attributes ||= begin
            if superclass.respond_to?(:attributes)
              superclass.attributes.clone
            else
              []
            end
          end
        end
        
        def required_attributes
          attributes.find_all { |attribute| !attribute.has_key?(:default) }
        end
        
        def properties
          attributes.find_all { |attribute| attribute[:type] == :property }
        end
        
        [:child, :children, :property].each do |attribute_type|
          class_eval <<-CODE
            def #{attribute_type}(name, options = {})
              class_eval { attr_reader name }
              attributes << options.merge(:type => :#{attribute_type}, :name => name)
            end
          CODE
        end
      end
      
      attr_reader :runtime, :location
      
      extend Forwardable
      def_delegators :runtime, :constants, :stack, :current_object, :current_location,
                               :current_scope, :current_failure_continuation
      
      def initialize(location = nil, *attributes)
        @location = location
        
        if self.class.required_attributes.length > attributes.length
          raise ArgumentError, "wrong number of attributes"
        end
        
        self.class.attributes.each do |attribute|
          instance_variable_set("@#{attribute[:name]}", attributes.shift || attribute[:default])
        end
      end
      
      def runtime=(runtime_object)
        @runtime = runtime_object
        children.compact.each { |child| child.runtime = runtime_object if child.is_a?(Node) }
      end
      
      def children
        @children ||= self.class.attributes.inject([]) do |children, attribute|
          value = instance_variable_get("@#{attribute[:name]}")
          
          if attribute[:type] == :children
            children + value
          elsif attribute[:type] == :child
            children << value
          else
            children
          end
        end
      end
      
      def eval_in_scope(scope, &continuation)
        eval_in_frame(Carat::Runtime::Frame.new(scope), &continuation)
      end
      
      def eval_with_failure_continuation(failure_continuation, &continuation)
        eval_in_frame(Carat::Runtime::Frame.new(nil, nil, failure_continuation), &continuation)
      end
      
      def eval_in_frame(frame, &continuation)
        stack << frame
        
        eval do |result|
          stack.pop
          yield result
        end
      end
      
      def eval_child(node, scope_or_failure_continuation = nil, &continuation)
        if node.nil?
          yield runtime.nil
        else
          if scope_or_failure_continuation
            case scope_or_failure_continuation
              when Carat::Runtime::Scope
                node.eval_in_scope(scope_or_failure_continuation, &continuation)
              when Proc
                node.eval_with_failure_continuation(scope_or_failure_continuation, &continuation)
            end
          else
            node.eval(&continuation)
          end
        end
      end
      
      def eval
        raise CaratError, "evaluation logic for #{self} not implemented"
      end
      
      def inspect
        Printer.new.print(self)
      end
      
      def to_ast
        self
      end
    end
    
    # A node which has a given single value when evaluated
    class ValueNode < Node
      def value_object
        raise NotImplementedError
      end
      
      def eval
        yield value_object
      end
    end
    
    # A node representing a value drawn from a set of possibilities - for example a string or
    # integer value
    class MultipleValueNode < ValueNode
      property :value
    end
    
    class NamedNode < Node
      property :name
    end
    
    class NodeList < Node
      children :items, :default => []
      
      def empty?
        items.empty?
      end
      
      # Fold the items by evaluating each one in turn and then passing the evaluated object to an
      # operation function
      def eval_fold(base_answer, operation, items = self.items, &continuation)
        # This lambda evaluates the AST node it is passed, and then computes the next answer for the
        # fold by combining the result with the current answer, using the operation provided, which
        # then yields to the fold_continuation
        fold_operation = lambda do |node, current_answer, &fold_continuation|
          eval_child(node) do |result|
            operation.call(result, current_answer, node, &fold_continuation)
          end
        end
        
        runtime.fold(base_answer, fold_operation, items, &continuation)
      end
    end
    
    class BinaryNode < Node
      child :left
      child :right
    end
    
    # ***** CONCRETE CLASSES ***** #
    
    require AST_PATH + "/scopes"
    require AST_PATH + "/messages"
    require AST_PATH + "/literals"
    require AST_PATH + "/variables"
    require AST_PATH + "/control"
  end
end

\end{lstlisting}
\begin{lstlisting}[title={\small\ttfamily\bfseries ast/control.rb},language=Ruby]
module Carat::AST
  class If < Node
    child :condition
    child :true_node
    child :false_node
    
    def eval(&continuation)
      eval_child(condition) do |condition_value|
        if condition_value.false_or_nil?
          eval_child(false_node, &continuation)
        else
          eval_child(true_node, &continuation)
        end
      end
    end
  end
  
  class While < Node
    child :condition
    child :contents
    
    def eval(&continuation)
      loop = lambda do
        eval_child(condition) do |condition_value|
          if condition_value.false_or_nil?
            yield runtime.nil
          else
            eval_child(contents) do |contents_value|
              loop.call
            end
          end
        end
      end
      
      loop.call
    end
  end
  
  class Begin < Node
    child :contents
    child :rescue
    
    def eval(&continuation)
      failure_continuation = self.rescue.failure_continuation(&continuation) if self.rescue
      eval_child(contents, failure_continuation, &continuation)
    end
  end
  
  class Rescue < Node
    child :error_type
    child :exception_variable
    child :contents
    
    def eval_error_type(&continuation)
      if error_type
        eval_child(error_type, &continuation)
      else
        yield constants[:RuntimeError]
      end
    end
    
    def check_error_type(exception, &continuation)
      eval_error_type do |error_type_object|
        exception.call(:is_a?, [error_type_object]) do |exception_match|
          yield exception_match == runtime.true
        end
      end
    end
    
    def assign_exception_variable(exception, &continuation)
      if exception_variable
        exception_variable.assign(exception, &continuation)
      else
        yield
      end
    end
    
    def failure_continuation(&continuation)
      lambda do |exception|
        # Remove the frame for this failure continuation from the stack
        stack.pop
        
        # If this failure continuation matches the error, evaluate its contents. Otherwise, unwind
        # the stack to the frame of the next failure continuation, and call that.
        check_error_type(exception) do |error_type_matches|
          if error_type_matches
            assign_exception_variable(exception) do
              eval_child(contents, &continuation)
            end
          else
            stack.unwind_to(:failure_continuation)
            current_failure_continuation.call(exception)
          end
        end
      end
    end
  end
  
  class And < BinaryNode
    def eval(&continuation)
      eval_child(left) do |left_value|
        if left_value.false_or_nil?
          yield left_value
        else
          eval_child(right, &continuation)
        end
      end
    end
  end
  
  class Or < BinaryNode
    def eval(&continuation)
      eval_child(left) do |left_value|
        if left_value.false_or_nil?
          eval_child(right, &continuation)
        else
          yield left_value
        end
      end
    end
  end
end

\end{lstlisting}
\begin{lstlisting}[title={\small\ttfamily\bfseries ast/literals.rb},language=Ruby]
module Carat::AST
  class True < ValueNode
    def value_object
      runtime.true
    end
  end
  
  class False < ValueNode
    def value_object
      runtime.false
    end
  end
  
  class Nil < ValueNode
    def value_object
      runtime.nil
    end
  end
  
  class String < MultipleValueNode
    def value_object
      constants[:String].new(value)
    end
  end
  
  class Integer < MultipleValueNode
    def value_object
      constants[:Fixnum].get(value)
    end
  end
  
  class Array < NodeList
    def eval(&continuation)
      append = lambda do |object, array_object, node, &append_continuation|
        append_continuation.call(array_object << object)
      end
      
      eval_fold([], append) do |item_objects|
        yield constants[:Array].new(item_objects)
      end
    end
  end
end

\end{lstlisting}
\begin{lstlisting}[title={\small\ttfamily\bfseries ast/messages.rb},language=Ruby]
module Carat::AST
  class MethodCall < Node
    child :receiver
    property :name
    child :arguments
    
    def eval_receiver(&continuation)
      if receiver
        eval_child(receiver, &continuation)
      else
        yield current_object
      end
    end
    
    def call(method_name, arguments, &continuation)
      eval_receiver do |receiver_object|
        method = receiver_object.lookup_instance_method(method_name)
        
        if method
          receiver_object.call(method, arguments, location, &continuation)
        else
          runtime.raise :NoMethodError, "undefined method '#{method_name}' for object #{receiver_object}", location
        end
      end
    end
    
    def eval(&continuation)
      call(name, arguments, &continuation)
    end
    
    def assign(value, &continuation)
      assign_arguments = Carat::AST::ArgumentList.new(
        location, arguments.items + [
          Carat::AST::ArgumentList::Item.new(location, value)
        ]
      )
      assign_arguments.runtime = runtime
      
      call("#{name}=".to_sym, assign_arguments, &continuation)
    end
  end
  
  class ArgumentList < NodeList
    class Item < Node
      child    :expression
      property :type, :default => :normal
      
      def eval(&continuation)
        if expression.is_a?(Node)
          eval_child(expression, &continuation)
        else
          yield expression
        end
      end
    end
    
    def eval(&continuation)
      append = lambda do |object, arguments, node, &append_continuation|
        case node.type
          when :splat
            object.call(:to_a) do |object_as_array|
              arguments.values += object_as_array.contents
              append_continuation.call(arguments)
            end
          when :block, :block_pass
            arguments.block = object
            append_continuation.call(arguments)
          else
            arguments.values << object
            append_continuation.call(arguments)
        end
      end
      
      eval_fold(Carat::Runtime::Arguments.new, append, &continuation)
    end
  end
  
  # This is a literal block, i.e. "foo do .. end" or "foo { ... }"
  # When evaluated it is converted to a lambda
  class Block < Node
    child :argument_pattern
    child :contents
    
    def eval
      yield constants[:Lambda].new(argument_pattern, contents, current_scope)
    end
  end
end

\end{lstlisting}
\begin{lstlisting}[title={\small\ttfamily\bfseries ast/printer.rb},language=Ruby]
module Carat::AST
  class Printer
    def initialize
      @indent = 0
    end
    
    def print(root_node)
      print_node(root_node)
    end
    
    private
    
      def print_node(node)
        return indent + "nil" if node.nil?
        
        result = indent + header(node)
        
        unless node.children.empty?
          result << ":\n"
          
          @indent += 1
          result << node.class.attributes.inject([]) do |items, attribute|
            if attribute[:type] == :child
              item = indent + attribute[:name].to_s + ":\n"
              @indent += 1
              item << print_node(node.send(attribute[:name]))
              @indent -= 1
              items << item
            elsif attribute[:type] == :children
              node.send(attribute[:name]).each do |child|
                items << print_node(child)
              end
            end
            
            items
          end.join("\n")
          @indent -= 1
        end
        
        result
      end
      
      def indent
        "  " * @indent
      end
      
      def header(node)
        header = node.class.to_s.sub("Carat::AST::", "")
        unless node.class.properties.empty?
          header << "["
          header << node.class.properties.map do |property|
            node.send(property[:name]).inspect
          end.join(", ")
          header << "]"
        end
        header
      end
  end
end

\end{lstlisting}
\begin{lstlisting}[title={\small\ttfamily\bfseries ast/scopes.rb},language=Ruby]
module Carat::AST
  class ExpressionList < NodeList
    def eval(&continuation)
      operation = lambda do |object, accumulation, node, &operation_continuation|
        operation_continuation.call(object)
      end
      
      eval_fold(runtime.nil, operation, &continuation)
    end
  end
  
  class ModuleDefinition < Node
    property :name
    child    :contents
    
    def module_object
      constants[name] ||= constants[:Module].new(name)
    end
    
    def contents_scope
      Carat::Runtime::Scope.new(module_object)
    end
    
    def eval(&continuation)
      eval_child(contents, contents_scope, &continuation)
    end
  end
  
  class ClassDefinition < Node
    property :name
    child    :superclass
    child    :contents
    
    def eval_superclass_object(&continuation)
      if superclass
        eval_child(superclass, &continuation)
      else
        yield constants[:Object]
      end
    end
    
    def eval_class_object
      eval_superclass_object do |superclass_object|
        yield constants[name] ||= constants[:Class].new(superclass_object, name)
      end
    end
    
    def eval_contents_scope
      eval_class_object do |class_object|
        yield Carat::Runtime::Scope.new(class_object)
      end
    end
    
    def eval(&continuation)
      eval_contents_scope do |contents_scope|
        eval_child(contents, contents_scope, &continuation)
      end
    end
  end
  
  class MethodDefinition < Node
    child    :receiver
    property :name
    child    :argument_pattern
    child    :contents
    
    def method_object
      constants[:Method].new(name, argument_pattern, contents)
    end
    
    def current_klass
      # If the current object is not a module or class (i.e. it is a normal object), get its class
      # (this could happen, for example, if a method is defined within another method)
      if current_object.is_a?(Carat::Data::ModuleInstance)
        current_object
      else
        current_object.real_klass
      end
    end
    
    def eval_klass(&continuation)
      if receiver
        # If there is a receiver this is a singleton method definition, so the method should
        # be placed in the method table of the singleton class of the receiver
        eval_child(receiver) do |receiver_object|
          yield receiver_object.singleton_class
        end
      else
        # Otherwise get the class in the current scope
        yield current_klass
      end
    end
    
    # Define a method in the current scope
    def eval
      eval_klass do |klass|
        klass.method_table[name] = method_object
        yield runtime.nil
      end
    end
  end
  
  class ArgumentPattern < NodeList
    class Item < Node
      child    :assignee
      property :type,    :default => :normal
      child    :default, :default => nil
      
      # A splat it considered mandatory, but it can match 0 arguments
      # A block pass is always optional and will default to nil
      def mandatory?
        type == :splat || (type == :normal && default.nil?)
      end
      
      def optional?
        !mandatory?
      end
      
      def minimum_arity
        case type
          when :splat, :block_pass
            0
          else
            default ? 0 : 1
        end
      end
      
      def maximum_arity
        case type
          when :splat
            (1.0/0) # Infinity
          when :block_pass
            0 # Block pass is not considered party of the arity
          else
            1
        end
      end
    end
    
    def minimum_arity
      items.inject(0) { |sum, item| sum + item.minimum_arity }
    end
    
    def maximum_arity
      items.inject(0) { |sum, item| sum + item.maximum_arity }
    end
    
    def arity
      @arity ||= minimum_arity..maximum_arity
    end
    
    def arity_as_string
      if minimum_arity == maximum_arity
        minimum_arity.to_s
      else
        "#{minimum_arity} to #{maximum_arity}"
      end
    end
    
    def normal_items_after_splat
      items.drop_while { |item| item.type != :splat }.
        drop(1).reject { |item| item.type == :block_pass}
    end
    
    def values_for_splat(values)
      values.shift(values.length - normal_items_after_splat.length)
    end
    
    def value_for(item, values, &continuation)
      case item.type
        when :splat
          yield runtime.constants[:Array].new(values_for_splat(values))
        when :block_pass
          yield current_scope.block || runtime.nil
        else
          value = values.shift
          
          if value
            yield value
          else
            eval_child(item.default, &continuation)
          end
      end
    end
    
    def assign(values, &continuation)
      if arity.include?(values.length)
        assign_item_operation = lambda do |item, &each_continuation|
          value_for(item, values) do |value|
            item.assignee.assign(value) do
              each_continuation.call
            end
          end
        end
        
        runtime.each(assign_item_operation, items, &continuation)
      else
        runtime.raise :ArgumentError, "wrong number of arguments (#{values.length} supplied, " +
                                      "#{arity_as_string} required)"
      end
    end
  end
end

\end{lstlisting}
\begin{lstlisting}[title={\small\ttfamily\bfseries ast/variables.rb},language=Ruby]
module Carat::AST
  class Assignment < Node
    child :receiver
    child :value
    
    # The receiver might be a local variable, instance variable, or method call. If it is a method
    # call then the value is technically an argument to the call, so we don't want to evaluate
    # it at this stage.
    def eval(&continuation)
      if receiver.is_a?(MethodCall)
        receiver.assign(value, &continuation)
      else
        eval_child(value) do |value_object|
          receiver.assign(value_object, &continuation)
        end
      end
    end
  end
  
  class LocalVariable < NamedNode
    def assign(value)
      yield current_scope[name] = value
    end
    
    # The only time when a local variable is explicitly distinguished from a method call is when we
    # have a line such as "foo ||= 42". In this case, the LHS is taken to be a local variable (not
    # a method call), and the expression is expanded into an AST node representing "foo = foo || 42",
    # but the occurance of "foo" on the RHS is also assumed to be a local variable.
    def eval(&continuation)
      if current_scope[name]
        yield current_scope[name]
      else
        runtime.raise :NameError, "undefined local variable '#{name}'"
      end
    end
  end
  
  class LocalVariableOrMethodCall < NamedNode
    def eval(&continuation)
      if current_scope[name]
        yield current_scope[name]
      elsif current_object.has_instance_method?(name)
        current_object.call(name, [], location, &continuation)
      else
        runtime.raise :NameError, "undefined local variable or method '#{name}'"
      end
    end
  end
  
  class InstanceVariable < NamedNode
    def assign(value)
      yield current_object.instance_variables[name] = value
    end
    
    def eval
      yield(current_object.instance_variables[name] || runtime.nil)
    end
  end
  
  class Constant < NamedNode
    def eval
      if constants[name]
        yield constants[name]
      else
        runtime.raise :NameError, "undefined constant '#{name}'"
      end
    end
  end
end

\end{lstlisting}
\begin{lstlisting}[title={\small\ttfamily\bfseries carat.rb},language=Ruby]
require "forwardable"

module Carat
  ROOT_PATH    = File.expand_path(File.dirname(__FILE__))
  RUNTIME_PATH = ROOT_PATH + "/runtime"
  DATA_PATH    = ROOT_PATH + "/data"
  KERNEL_PATH  = ROOT_PATH + "/kernel"
  AST_PATH     = ROOT_PATH + "/ast"
  PARSER_PATH  = ROOT_PATH + "/parser"
  
  class CaratError < StandardError; end
  
  require DATA_PATH    + "/data"
  require RUNTIME_PATH + "/runtime"
  require AST_PATH     + "/ast"
  require PARSER_PATH  + "/parser"
  
  class Location
    attr_reader :file_name, :line, :column
    
    def initialize(file_name, line, column)
      @file_name, @line, @column = file_name, line, column
    end
    
    def to_s
      "#{file_name} at line #{line}, col #{column}"
    end
  end
  
  def self.parse(input, file_name = nil)
    LanguageParser.new(input, file_name).ast
  end
  
  def self.run(input)
    Runtime.new.run(input)
  end
  
  def self.run_file(name)
    Runtime.new.run_file(name)
  end
end

\end{lstlisting}
\begin{lstlisting}[title={\small\ttfamily\bfseries data/array.rb},language=Ruby]
module Carat::Data
  class ArrayClass < ClassInstance
    def new(items)
      ArrayInstance.new(runtime, self, items)
    end
  end
  
  class ArrayInstance < ObjectInstance
    attr_reader :contents
    
    def initialize(runtime, klass, contents = [])
      super(runtime, klass)
      @contents = contents
    end
    
    def primitive_initialize(*contents)
      @contents = contents
      yield runtime.nil
    end
    
    def primitive_length
      yield constants[:Fixnum].get(@contents.length)
    end
    
    def primitive_each(block, &continuation)
      yield_operation = lambda do |item, &each_continuation|
        block.call(:call, [item], &each_continuation)
      end
      
      runtime.each(yield_operation, @contents, self, &continuation)
    end
    
    def primitive_push(item)
      @contents << item
      yield self
    end
    
    def primitive_get(i)
      yield @contents[i.value] || runtime.nil
    end
    
    def primitive_set(i, value)
      yield @contents[i.value] = value
    end
  end
end

\end{lstlisting}
\begin{lstlisting}[title={\small\ttfamily\bfseries data/class.rb},language=Ruby]
module Carat::Data
  class ClassClass < ModuleClass
    def new(superclass, name = nil)
      ClassInstance.new(runtime, self, superclass, name)
    end
  end
  
  class ClassInstance < ModuleInstance
    attr_accessor :super
    
    def initialize(runtime, klass, superclass, name = nil)
      @super = superclass
      super(runtime, klass, name || inferred_name)
    end
    
    def lookup_method(name)
      method_table[name] || (@super && @super.lookup_method(name))
    end
    
    def ancestors
      if @super
        [self] + @super.ancestors
      else
        [self]
      end
    end
    
    # The super may be an include class, so we want the first ancestor which is a "proper" class
    def superclass
      if @super.is_a?(IncludeClassInstance)
        @super && @super.superclass
      else
        @super
      end
    end
    
    def insert_include_class(mod)
      @super = IncludeClassInstance.new(runtime, mod, @super)
    end
    
    def to_s
      "<class:#{name}>"
    end
    
    private
    
      def create_singleton_class
        if constants[:SingletonClass]
          self.klass = constants[:SingletonClass].new(self, superclass && superclass.singleton_class)
        end
      end
      
      # The inferred name is the name of the class in the object language, taken from the name of the
      # class representing it in the implementation language. For instance, if this is an instance of
      # +FixnumClass+, then the inferred name is +:Fixnum+
      def inferred_name
        @inferred_name ||= begin
          # We must be in a subclass of ClassInstance in order to infer a name
          unless instance_of?(ClassInstance)
            self.class.to_s.sub(/^.*\:\:/, '').sub(/Class$/, '').to_sym
          end
        end
      end
      
      # Returns the class which is used to represent an instance of this class.
      # 
      # For example, if this class is +FixnumClass+, the +instance_class+ will be +FixnumInstance+
      def instance_class
        @instance_class ||= begin
          ancestors.each do |ancestor|
            if Carat::Data.const_defined?("#{ancestor.name}Instance")
              return Carat::Data.const_get("#{ancestor.name}Instance")
            end
          end
        end
      end
    
    public
    
    # ***** Primitives ***** #
    
    def primitive_allocate
      yield instance_class.new(runtime, self)
    end
    
    def primitive_superclass
      yield superclass || runtime.nil
    end
    
    def primitive_include(mod)
      instance_class.send(:include, mod.primitives_module) if mod.primitives_module
      
      insert_include_class(mod)
      singleton_class.insert_include_class(mod.singleton_class)
      
      yield mod
    end
  end
end

\end{lstlisting}
\begin{lstlisting}[title={\small\ttfamily\bfseries data/data.rb},language=Ruby]
module Carat
  module Data
    # First, very clearly specify the basic hierarchy of data classes. This mirrors the inheritance
    # hierarchy in the source language:
    # 
    #   class Object         < nil;    end
    #   class Module         < Object; end
    #   class Class          < Module; end
    #   class SingletonClass < Class;  end
    # 
    class ObjectInstance; end
    
    class ModuleInstance         < ObjectInstance; end
    class ClassInstance          < ModuleInstance; end
    class SingletonClassInstance < ClassInstance;  end
    
    class ObjectClass            < ClassInstance;  end
    class ModuleClass            < ObjectClass;    end
    class ClassClass             < ModuleClass;    end
    class SingletonClassClass    < ClassClass;     end
    
    # Now, require the actual code
    require DATA_PATH + '/kernel'
    require DATA_PATH + '/object'
    require DATA_PATH + '/module'
    require DATA_PATH + '/class'
    
    require DATA_PATH + '/singleton_class'
    require DATA_PATH + '/include_class'
    
    require DATA_PATH + '/lambda'
    require DATA_PATH + '/method'
    require DATA_PATH + '/primitive'
    require DATA_PATH + '/exception'
    
    require DATA_PATH + '/fixnum'
    require DATA_PATH + '/array'
    require DATA_PATH + '/string'
    require DATA_PATH + '/singletons'
  end
end

\end{lstlisting}
\begin{lstlisting}[title={\small\ttfamily\bfseries data/exception.rb},language=Ruby]
module Carat::Data
  class ExceptionClass < ClassInstance
  end
  
  class ExceptionInstance < ObjectInstance
    attr_reader :backtrace
    
    def generate_backtrace(location)
      locations       = [location] + call_stack.reverse.map(&:location)
      enclosing_calls = call_stack.reverse + [nil]
      backtrace       = locations.zip(enclosing_calls)[0..-2]
      
      @backtrace = backtrace.map do |location, enclosing_call|
        "#{location} in #{enclosing_call}"
      end
    end
    
    def primitive_backtrace
      backtrace = @backtrace.map { |line| constants[:String].new(line) }
      yield constants[:Array].new(backtrace)
    end
  end
end

\end{lstlisting}
\begin{lstlisting}[title={\small\ttfamily\bfseries data/fixnum.rb},language=Ruby]
module Carat::Data
  class FixnumClass < ClassInstance
    def instances
      @instances ||= {}
    end
    
    def get(number)
      instances[number] ||= FixnumInstance.new(runtime, self, number)
    end
  end
  
  class FixnumInstance < ObjectInstance
    attr_reader :value
    
    def initialize(runtime, klass, value)
      @value = value
      super(runtime, klass)
    end
    
    def to_s
      value && value.to_s || super
    end
    
    # ***** Primitives ***** #
    
    def primitive_spaceship(other)
      yield klass.get(value <=> other.value)
    end
    
    def primitive_plus(other)
      yield klass.get(value + other.value)
    end
    
    def primitive_minus(other)
      yield klass.get(value - other.value)
    end
    
    def primitive_multiply(other)
      yield klass.get(value * other.value)
    end
    
    def primitive_divide(other)
      yield klass.get(value / other.value)
    end
    
    def primitive_to_s
      yield constants[:String].new(value.to_s)
    end
  end
end

\end{lstlisting}
\begin{lstlisting}[title={\small\ttfamily\bfseries data/include\_class.rb},language=Ruby]
module Carat::Data
  class IncludeClassInstance < ClassInstance
    attr_reader :module
    
    extend Forwardable
    def_delegators :"self.module", :primitives_module, :extensions_module,
                                   :lookup_instance_method, :name
    
    def initialize(runtime, mod, supr)
      @module = mod
      super(runtime, mod, supr)
      
      # An include class does not have its own method table, it uses the method table of the module
      # being included
      @method_table = mod.method_table
    end
    
    def to_s
      "<include_class:#{klass}>"
    end
  end
end

\end{lstlisting}
\begin{lstlisting}[title={\small\ttfamily\bfseries data/kernel.rb},language=Ruby]
module Carat::Data
  module KernelModule
    def primitive_puts(object)
      if object == runtime.nil
        Kernel.puts("nil")
        yield runtime.nil
      else
        # object.call(:to_s) gets the StringInstance representing the object, and then calling
        # to_s actually gets the string.
        object.call(:to_s) do |object_as_string|
          Kernel.puts(object_as_string.to_s)
          yield runtime.nil
        end
      end
    end
    
    # Yield the caller's current block
    def primitive_yield(*args, &continuation)
      block = current_call.caller_scope.block
      
      if block
        block.primitive_call(*args, &continuation)
      else
        runtime.raise :ArgumentError, "no block given"
      end
    end
    
    # Throw away the current continuation and call the failure continuation
    def primitive_raise(exception, &continuation)
      # Store the location of the call to Kernel#raise
      location = current_location
      
      # Remove the frame for the Kernel#raise call
      stack.pop
      
      # Generate the exception's backtrace before we modify the stack
      exception.generate_backtrace(location)
      
      # Unwind the stack until we get to a failure continuation
      stack.unwind_to(:failure_continuation)
      
      # Call the failure continuation which is now at the top of the stack
      current_failure_continuation.call(exception)
    end
    
    # Return from a method on the call stack without doing any further computation
    def primitive_return(value, &continuation)
      # Remove the frame for the Kernel#return call
      stack.pop
      
      # Unwind the stack until we get to a call
      stack.unwind_to(:call)
      
      # Call the return continuation of the current call (which will take care of popping the call
      # off the stack)
      current_call.return_continuation.call(value)
    end
    
    def primitive_require(file, &continuation)
      file_location = File.dirname(current_location.file_name) + "/" + file.to_s
      
      if runtime.loaded_files.include?(file_location)
        yield runtime.false
      else
        runtime.run_file(file_location + ".carat")
        yield runtime.true
      end
    end
  end
end

\end{lstlisting}
\begin{lstlisting}[title={\small\ttfamily\bfseries data/lambda.rb},language=Ruby]
module Carat::Data
  class LambdaClass < ClassInstance
    def new(argument_pattern, contents, scope)
      LambdaInstance.new(runtime, self, argument_pattern, contents, scope)
    end
  end
  
  class LambdaInstance < ObjectInstance
    attr_reader :argument_pattern, :contents, :scope
    
    def initialize(runtime, klass, argument_pattern, contents, scope)
      @argument_pattern, @contents, @scope = argument_pattern, contents, scope
      super(runtime, klass)
    end
    
    # Extend the scope in which the block was created. The reason for extending the scope is that 
    # it means any fresh variables within the lambda will stay local to the lambda.
    def evaluation_scope
      scope.extend
    end
    
    def to_s
      "<lambda>"
    end
    
    ##### PRIMITIVES #####
    
    def primitive_call(*arguments, &continuation)
      arguments = Carat::Runtime::Arguments.from_a(arguments)
      
      Carat::Runtime::Call.new(
        runtime, self, arguments,
        continuation, evaluation_scope, current_location
      ).send
    end
  end
end

\end{lstlisting}
\begin{lstlisting}[title={\small\ttfamily\bfseries data/method.rb},language=Ruby]
module Carat::Data
  class MethodClass < ClassInstance
    def new(name, argument_pattern, contents)
      MethodInstance.new(runtime, self, name, argument_pattern, contents)
    end
  end
  
  class MethodInstance < ObjectInstance
    attr_reader :name, :argument_pattern, :contents
    
    def initialize(runtime, klass, name, argument_pattern, contents)
      @name, @argument_pattern, @contents = name, argument_pattern, contents
      super(runtime, klass)
    end
    
    def to_s
      "<method:#{name}>"
    end
  end
end

\end{lstlisting}
\begin{lstlisting}[title={\small\ttfamily\bfseries data/module.rb},language=Ruby]
module Carat::Data
  class ModuleClass < ObjectClass
    def new(name)
      ModuleInstance.new(runtime, self, name)
    end
  end
  
  class ModuleInstance < ObjectInstance
    attr_reader   :name, :method_table
    
    def initialize(runtime, klass, name = nil)
      @name         = name
      @method_table = {}
      
      super(runtime, klass)
      
      include_module_primitives if include_class?
      create_singleton_class unless include_class? || singleton?
    end
    
    def singleton?
      instance_of?(SingletonClassInstance)
    end
    
    def include_class?
      instance_of?(IncludeClassInstance)
    end
    
    # If this is actually a module (as opposed to a class or whatever) then we can have a module
    # in the implementation language containing primitives for this specific module in the source
    # language.
    # 
    # For instance, if we create a +ModuleInstance+ with name "Kernel", then the module named
    # "KernelModule", defined primitives for it.
    # 
    # This is useful, because then when "Kernel" is included in another module/class, we can also
    # make the primitives available to the module/class it is included in. 
    def primitives_module
      if name && Carat::Data.const_defined?("#{name}Module")
        Carat::Data.const_get("#{name}Module")
      end
    end
    
    def to_s
      "<module:#{name}>"
    end
    
    private
      
      def include_module_primitives
        extend(primitives_module) if primitives_module
      end
    
    public
    
    # ***** Primitives ***** #
    
    def primitive_name
      yield constants[:String].new(name)
    end
  end
end

\end{lstlisting}
\begin{lstlisting}[title={\small\ttfamily\bfseries data/object.rb},language=Ruby]
module Carat::Data
  class ObjectClass < ClassInstance
    def new
      ObjectInstance.new(runtime, self)
    end
  end
  
  class ObjectInstance
    class << self
      def next_object_id
        if @current_object_id
          @current_object_id += 1
        else
          @current_object_id = 1
        end
      end
    end
    
    attr_reader :runtime, :carat_object_id, :instance_variables
    attr_accessor :klass
    
    extend Forwardable
    def_delegators :runtime, :constants, :stack, :current_location, :current_failure_continuation,
                   :current_call, :current_scope, :current_object, :call_stack
    
    def initialize(runtime, klass)
      @runtime, @klass    = runtime, klass
      @carat_object_id    = ObjectInstance.next_object_id
      @instance_variables = {}
    end
    
    # Lookup a instance method - i.e. one defined by this object's class
    def lookup_instance_method(name)
      klass.lookup_method(name)
    end
    
    # Lookup an intance method or raise an exception
    def lookup_instance_method!(name)
      lookup_instance_method(name) || raise(Carat::CaratError, "undefined method '#{name}'")
    end
    
    def has_instance_method?(name)
      lookup_instance_method(name) != nil
    end
    
    # Call the method with a given name, with the given argument list (AST::ArgumentList or Array).
    # This should only be called when we know the method exists. If the method does not exist an
    # exception will be raised.
    def call(method_or_name, argument_list = [], location = current_location, &continuation)
      if method_or_name.is_a?(Symbol)
        method = lookup_instance_method!(method_or_name)
      else
        method = method_or_name
      end
      
      create_call(method, argument_list, location, continuation).send
    end
    
    def singleton_class
      klass && klass.singleton? ? klass : create_singleton_class
    end
    
    # A 'real class' is the first one in the ancestry of the actual class, which is not a singleton
    def real_klass
      if klass
        real_klass = klass
        
        while real_klass && real_klass.singleton?
          real_klass = real_klass.superclass
        end
        
        real_klass
      end
    end
    
    def false_or_nil?
      instance_of?(FalseClassInstance) || instance_of?(NilClassInstance)
    end
    
    def to_s
      inspect
    end
    
    def inspect
      "<object:#{klass}>"
    end
    
    private
    
      def create_call(method, argument_list, location, continuation)
        Carat::Runtime::Call.new(
          runtime, method, argument_list,
          continuation, method_scope, location
        )
      end
      
      # A scope for evaluating the method call, with this object as 'self'
      def method_scope
        Carat::Runtime::Scope.new(self)
      end
      
      def create_singleton_class
        self.klass = constants[:SingletonClass].new(self, klass)
      end
    
    public
    
    # ***** Primitives ***** #
    
    def primitive_equal_to(other)
      if carat_object_id == other.carat_object_id
        yield runtime.true
      else
        yield runtime.false
      end
    end
    
    def primitive_object_id
      yield constants[:Fixnum].get(carat_object_id)
    end
    
    def primitive_class
      yield real_klass
    end
  end
end

\end{lstlisting}
\begin{lstlisting}[title={\small\ttfamily\bfseries data/primitive.rb},language=Ruby]
module Carat::Data
  class PrimitiveClass < ClassInstance
    def lookup_instance_method(name)
      primitive_name = "primitive_#{name}"
      current_object.method(primitive_name) if current_object.respond_to?(primitive_name)
    end
    
    private
    
      def create_call(method, argument_list, location, continuation)
        Carat::Runtime::PrimitiveCall.new(runtime, method, argument_list, continuation)
      end
  end
end

\end{lstlisting}
\begin{lstlisting}[title={\small\ttfamily\bfseries data/singleton\_class.rb},language=Ruby]
module Carat::Data
  class SingletonClassClass < ClassClass
    def new(owner, superclass)
      SingletonClassInstance.new(runtime, owner, superclass)
    end
  end
  
  class SingletonClassInstance < ClassInstance
    attr_reader :owner
    
    def initialize(runtime, owner, superclass)
      @owner = owner
      super(runtime, superclass && superclass.klass, superclass)
    end
    
    def to_s
      "<singleton_class:#{owner}>"
    end
  end
end

\end{lstlisting}
\begin{lstlisting}[title={\small\ttfamily\bfseries data/singletons.rb},language=Ruby]
module Carat::Data
  class SingletonObjectClass < ClassInstance
    def instance
      @instance ||= instance_class.new(runtime, self)
    end
  end
  
  class FalseClassClass < SingletonObjectClass; end
  class FalseClassInstance < ObjectInstance;    end
  class TrueClassClass < SingletonObjectClass;  end
  class TrueClassInstance < ObjectInstance;     end
  class NilClassClass < SingletonObjectClass;   end
  class NilClassInstance < ObjectInstance;      end
end

\end{lstlisting}
\begin{lstlisting}[title={\small\ttfamily\bfseries data/string.rb},language=Ruby]
module Carat::Data
  class StringClass < ClassInstance
    def new(contents = "")
      StringInstance.new(runtime, self, contents)
    end
  
    def primitive_allocate
      yield new
    end
  end
  
  class StringInstance < ObjectInstance
    attr_reader :contents
  
    def initialize(runtime, klass, contents = "")
      # clone the contents string because it is important to make sure that two separate 
      # StringInstances aren't stored by the same underlying String object
      @contents = contents.to_s.clone
      super(runtime, klass)
    end
    
    def to_s
      contents
    end
    
    # ***** Primitives ***** #
    
    def primitive_inspect
      yield real_klass.new(contents.inspect)
    end
    
    def primitive_plus(other)
      yield real_klass.new(contents + other.contents)
    end
    
    def primitive_push(other)
      contents << other.contents
      yield self
    end
    
    def primitive_equal_to(other)
      if contents == other.contents
        yield runtime.true
      else
        yield runtime.false
      end
    end
  end
end

\end{lstlisting}
\begin{lstlisting}[title={\small\ttfamily\bfseries kernel/array.carat},language=Ruby]
class Array
  def initialize(*contents)
    Primitive.initialize(*contents)
  end
  
  def length
    Primitive.length
  end
  
  def each(&block)
    Primitive.each(&block)
  end
  
  def <<(item)
    Primitive.push(item)
  end
  
  def [](index)
    Primitive.get(index)
  end
  
  def []=(index, value)
    Primitive.set(index, value)
  end
  
  def ==(other)
    if length != other.length
      return false
    end
    
    i = 0
    while i != length
      if self[i] != other[i]
        return false
      end
      i += 1
    end
    
    return true
  end
  
  def to_a
    self
  end
  
  def to_s
    map { |item| item.to_s }.join("\n")
  end
  
  def map
    ary = []
    each { |item| ary << yield(item) }
    ary
  end
  
  def inspect
    "[" + map { |item| item.inspect }.join(", ") + "]"
  end
  
  def join(joiner = "")
    result = ""
    i = 1
    each do |item|
      result << item.to_s
      
      if i != length
        result << joiner.to_s
      end
      
      i = i + 1
    end
    result
  end
end

\end{lstlisting}
\begin{lstlisting}[title={\small\ttfamily\bfseries kernel/class.carat},language=Ruby]
class Class < Module
  def allocate
    Primitive.allocate
  end
  
  def superclass
    Primitive.superclass
  end
  
  def include(mod)
    Primitive.include(mod)
  end
  
  def new(*args, &block)
    object = self.allocate
    object.initialize(*args, &block)
    object
  end
end

\end{lstlisting}
\begin{lstlisting}[title={\small\ttfamily\bfseries kernel/comparable.carat},language=Ruby]
module Comparable
  def <=>(other)
    raise RuntimeError.new("<=> not implemented")
  end
  
  def <(other)
    (self <=> other) == -1
  end
  
  def >(other)
    (self <=> other) == 1
  end
  
  def <=(other)
    self < other || self == other
  end
  
  def >=(other)
    self > other || self == other
  end
end

\end{lstlisting}
\begin{lstlisting}[title={\small\ttfamily\bfseries kernel/exception.carat},language=Ruby]
class Exception
  def initialize(message = "(no message)")
    @message = message
  end
  
  def to_s
    @message
  end
  
  def backtrace
    Primitive.backtrace
  end
end

class StandardError < Exception; end
class NameError < StandardError; end
class NoMethodError < NameError; end
class ArgumentError < StandardError; end
class RuntimeError < StandardError; end

\end{lstlisting}
\begin{lstlisting}[title={\small\ttfamily\bfseries kernel/false\_class.carat},language=Ruby]
class FalseClass
  def to_s
    "false"
  end
  
  def inspect
    to_s
  end
end

\end{lstlisting}
\begin{lstlisting}[title={\small\ttfamily\bfseries kernel/fixnum.carat},language=Ruby]
class Fixnum
  include Comparable
  
  def <=>(other)
    Primitive.spaceship(other)
  end
  
  def +(other)
    Primitive.plus(other)
  end
  
  def -(other)
    Primitive.minus(other)
  end
  
  def *(other)
    Primitive.multiply(other)
  end
  
  def /(other)
    Primitive.divide(other)
  end
  
  # Unary -
  def --
    0 - self
  end
  
  # Unary +
  def ++
    self
  end
  
  def to_s
    Primitive.to_s
  end
  
  def inspect
    to_s
  end
end

\end{lstlisting}
\begin{lstlisting}[title={\small\ttfamily\bfseries kernel/kernel.carat},language=Ruby]
module Kernel
  def raise(exception)
    Primitive.raise(exception)
  end
  
  def puts(obj = "\n")
    Primitive.puts(obj)
  end

  def p(obj)
    puts obj.inspect
  end
  
  def lambda(&block)
    Lambda.new(&block)
  end
  
  def yield(*args, &block)
    Primitive.yield(*args, &block)
  end
  
  def return(value = nil)
    Primitive.return(value)
  end
  
  def require(file)
    Primitive.require(file)
  end
end

\end{lstlisting}
\begin{lstlisting}[title={\small\ttfamily\bfseries kernel/lambda.carat},language=Ruby]
class Lambda
  def self.new(&block)
    block
  end
  
  def call(*args, &block)
    Primitive.call(*args, &block)
  end
end

\end{lstlisting}
\begin{lstlisting}[title={\small\ttfamily\bfseries kernel/module.carat},language=Ruby]
class Module
  def name
    Primitive.name
  end
  
  def inspect
    name
  end
  
  def to_s
    name
  end
end

\end{lstlisting}
\begin{lstlisting}[title={\small\ttfamily\bfseries kernel/nil\_class.carat},language=Ruby]
class NilClass
  def to_s
    ""
  end
  
  def inspect
    "nil"
  end
end

\end{lstlisting}
\begin{lstlisting}[title={\small\ttfamily\bfseries kernel/object.carat},language=Ruby]
class Object
  include Kernel
  
  def initialize
    # Do nothing by default
  end
  
  def ==(other)
    Primitive.equal_to(other)
  end
  
  def !=(other)
    if self == other
      false
    else
      true
    end
  end
  
  def !!
    if self == nil || self == false
      true
    else
      false
    end
  end
  
  def is_a?(test_class)
    klass = self.class
    
    while klass != nil
      if klass == test_class
        return true
      else
        klass = klass.superclass
      end
    end
    
    return false
  end
  
  def object_id
    Primitive.object_id
  end
  
  def class
    Primitive.class
  end
  
  def inspect
    "<" + self.class.to_s + ":" + object_id.to_s + ">"
  end
  
  def to_s
    inspect
  end
end

\end{lstlisting}
\begin{lstlisting}[title={\small\ttfamily\bfseries kernel/string.carat},language=Ruby]
class String
  def +(other)
    Primitive.plus(other.to_s)
  end
  
  def <<(other)
    Primitive.push(other.to_s)
  end
  
  def to_s
    self
  end
  
  def inspect
    Primitive.inspect
  end
end

\end{lstlisting}
\begin{lstlisting}[title={\small\ttfamily\bfseries kernel/true\_class.carat},language=Ruby]
class TrueClass
  def to_s
    "true"
  end
  
  def inspect
    to_s
  end
end

\end{lstlisting}
\begin{lstlisting}[title={\small\ttfamily\bfseries parser/comment.treetop},language=treetop]
module Carat
  # This is a simple parser which strips comments from the source code before that code is actually
  # parsed. It is simpler to keep this step separate.
  # 
  # There are two kinds of comment:
  # 
  #   1. Starts with '#' and finished with \n
  #   2. Starts with '##' and finishes with next occurrence of '##'
  # 
  # With the second kind, newlines are preserved when stripping, so that error message which
  # involve line numbers still make sense.
  grammar Comment
    rule program
      head:line tail:("\n" line)* "\n"? {
        def lines
          [head] + tail.elements.map(&:line)
        end
        
        def strip
          lines.map(&:strip).join("\n")
        end
      }
      /
      '' {
        def strip
          ''
        end
      }
    end
    
    rule line
      parts:(string / non_string)* comment:comment? {
        def stripped_comment
          comment.empty? ? '' : comment.strip
        end
      
        def strip
          parts.text_value + stripped_comment
        end
      }
    end
    
    rule non_string
      [^\#\n\"\']+
    end
    
    rule string
      '"' [^\"]* '"' /
      "'" [^\']* "'"
    end
    
    rule comment
      multi_line_comment /
      single_line_comment
    end
    
    rule multi_line_comment
      '##' (!'##' .)* '##' {
        def strip
          # Keep the new lines so that error messages which refer to a line number still make sense
          text_value.gsub(/[^\n]/, "")
        end
      }
    end
    
    rule single_line_comment
      '#' [^\n]* {
        def strip
          ''
        end
      }
    end
  end
end

\end{lstlisting}
\begin{lstlisting}[title={\small\ttfamily\bfseries parser/language.treetop},language=treetop]
module Carat
  grammar Language
    # A program is the top-level node, containing just a list of expressions
    rule program
      multiline_space? expression_list <Program>
    end
    
    # 1 or more expressions separated by terminators
    rule expression_list
      first:expression rest:(space? terminator expression)* space? terminator? <ExpressionList> /
      '' <EmptyExpressionList>
    end
    
    # An expression is the basic 'thing'
    rule expression
      assignment_expression
    end
    
    rule assignment_expression
      simple_assignment /
      binary_method_assignment /
      binary_operation_assignment /
      or_expression
    end
    
    rule assignee
      assignee_method_call_chain / variable
    end
    
    rule simple_assignment
      receiver:assignee space?
      '=' multiline_space?
      value:expression <Assignment>
    end
    
    rule binary_method_assignment
      receiver:assignee space?
      name:('<<' / '>>' / '+' / '-' / '*' / '/') '=' multiline_space?
      value:expression <BinaryMethodAssignment>
    end
    
    rule binary_operation_assignment
      receiver:assignee space?
      name:('||' / '&&') '=' multiline_space?
      value:expression <BinaryOperationAssignment>
    end
    
    rule or_expression
      left:and_expression space?
      name:'||' multiline_space?
      right:expression <BinaryOperation>
      /
      and_expression
    end
    
    rule and_expression
      left:comparison_expression space?
      name:'&&' multiline_space?
      right:expression <BinaryOperation>
      /
      comparison_expression
    end
    
    rule comparison_expression
      left:inequality_expression space?
      name:('==' / '!=' / '===' / '<=>') multiline_space?
      right:comparison_expression <BinaryMethodCall>
      /
      inequality_expression
    end
    
    rule inequality_expression
      left:shift_expression space?
      name:('<=' / '>=' / '<' / '>') multiline_space?
      right:inequality_expression <BinaryMethodCall>
      /
      shift_expression
    end
    
    rule shift_expression
      left:add_subtract_expression space?
      name:('<<' / '>>') multiline_space?
      right:shift_expression <BinaryMethodCall>
      /
      add_subtract_expression
    end
    
    rule add_subtract_expression
      left:times_divide_expression space?
      name:('+' / '-') multiline_space?
      right:add_subtract_expression <BinaryMethodCall>
      /
      times_divide_expression
    end
    
    rule times_divide_expression
      left:unary_not_expression space?
      name:('*' / '/') multiline_space?
      right:times_divide_expression <BinaryMethodCall>
      /
      unary_not_expression
    end
    
    rule unary_not_expression
      name:'!' multiline_space? receiver:unary_not_expression <UnaryMethodCall> /
      method_call_expression
    end
    
    rule method_call_expression
      method_call_chain / unary_plus_minus_expression
    end
    
    rule unary_plus_minus_expression
      name:('+' / '-') multiline_space? receiver:unary_plus_minus_expression <UnaryMethodCall> /
      primary
    end
    
    rule primary
      module_definition /
      class_definition /
      method_definition /
      control_structure /
      literal /
      instance_variable /
      constant /
      local_variable_or_method_call /
      bracketed_expression
    end
    
    rule bracketed_expression
      '(' expression ')' <BracketedExpression>
    end
    
    rule definition_body
      terminator expression_list 'end'
    end
    
    rule module_definition
      'module' multiline_space
      constant
      definition_body <ModuleDefinition>
    end
    
    rule class_definition
      'class' multiline_space
      constant space?
      superclass:('<' multiline_space? primary)?
      definition_body <ClassDefinition>
    end
    
    rule method_definition
      'def' multiline_space
      receiver:(primary space? '.' multiline_space?)?
      method_name method_argument_pattern
      definition_body <MethodDefinition>
    end
    
    # The arguments defined as part of a method definition
    rule method_argument_pattern
      space? '(' multiline_space?
      contents:argument_pattern_contents
      multiline_space? ')' <ArgumentPattern> /
      '' <ArgumentPattern>
    end
    
    # The arguments defined as part of a block definition
    rule block_argument_pattern
      space? '|' multiline_space?
      contents:argument_pattern_contents
      multiline_space? '|' <ArgumentPattern> /
      '' <ArgumentPattern>
    end
    
    rule argument_pattern_contents
      head:argument_pattern_item
      tail:(multiline_space? ',' multiline_space? item:argument_pattern_item)* /
      ''
    end
    
    rule argument_pattern_item
      assignee default:(
        multiline_space? '='
        multiline_space? expression
      )? <ArgumentPatternItem> /
      ('*' / '&') multiline_space? assignee <ArgumentPatternItem>
    end
    
    rule control_structure
      if_expression /
      while_expression /
      begin_expression
    end
    
    rule if_expression
      'if' multiline_space condition:expression space? terminator
      true_block:expression_list
      false_block:(nested_if_expression / else_expression)?
      'end' <IfExpression>
    end
    
    rule nested_if_expression
      'elsif' multiline_space condition:expression space? terminator
      true_block:expression_list
      false_block:(nested_if_expression / else_expression)? <IfExpression>
    end
    
    rule else_expression
      'else' multiline_space expression_list <ElseExpression>
    end
    
    rule while_expression
      'while' space condition:expression space? terminator
      contents:expression_list
      'end' <WhileExpression>
    end
    
    rule begin_expression
      'begin' multiline_space
      contents:expression_list
      rescue:rescue_expression?
      'end' <BeginExpression>
    end
    
    rule rescue_expression
      'rescue'
      type:(space expression)?
      assignment:(space? '=>' multiline_space variable)?
      multiline_space contents:expression_list <RescueExpression>
    end
    
    # A literal object, e.g. a number, string, true, false, etc
    rule literal
      number / array / string / boolean / nil
    end
    
    # A chain of one or more method calls. This matches only when we know for sure that a method is
    # being called. So for example just "foo" will not match - as that may be a local variable or
    # a method call. But "foo()" is definitely a method call, so we can match that. A chain is of 
    # the form "foo.bar(...).baz(...)" and we have to convert this to several MethodCall instances
    # when creating the AST.
    rule method_call_chain
      receiver:unary_plus_minus_expression tail:method_call_segment+ <MethodCallChain> /
      head:implicit_method_call            tail:method_call_segment* <ImplicitMethodCallChain>
    end
    
    # A method call chain which can be assigned to. This means that the last call in the chain
    # must have a 'simple' name with no special characters.
    rule assignee_method_call_chain
      receiver:unary_plus_minus_expression
      middle:(method_call_segment &'.')*
      last:assignee_method_call_segment <AssigneeMethodCallChain> /
      
      head:implicit_method_call
      middle:(method_call_segment &'.')*
      last:assignee_method_call_segment <ImplicitAssigneeMethodCallChain>
    end
    
    # If the method name matches an identifier, there has to be some sort of recognisable argument
    # list for us to be sure it is a method call. Otherwise, the method name is clearly using some
    # characters which are specific to methods (for example a '?' at the end), so the argument
    # list is optional.
    rule implicit_method_call
      method_name:identifier           argument_list /
      method_name:implicit_method_name argument_list:argument_list?
    end
    
    rule method_call_segment
      item:(dot_method_call / element_reference) space?
    end
    
    rule assignee_method_call_segment
      item:(assignee_dot_method_call / element_reference) space?
    end
    
    rule dot_method_call
      '.' multiline_space? method_name:method_name argument_list:argument_list?
    end
    
    rule assignee_dot_method_call
      '.' multiline_space? method_name:assignee_method_name
    end
    
    rule element_reference
      '[' multiline_space? ']' <ElementReference> /
      '[' multiline_space?
      head:argument_list_item
      tail:(multiline_space? ',' multiline_space? argument_list_item)*
      multiline_space? ']' <ElementReference>
    end
    
    # Argument list - values which are passed during a method call
    # This is when there are definitely arguments but they are possibly empty (i.e. an
    # empty pair of parentheses). If there is simply an empty string (i.e. an empty list of
    # arguments *not* surrounded by parentheses) these are not matched - this special case is dealt
    # with by the local_variable_or_method_call rule
    #
    # A block is also considered to be a part of the argument list
    rule argument_list
      bracketed_argument_list /
      unbracketed_argument_list
    end
    
    # Any number of items inside parentheses, with an optional block
    rule bracketed_argument_list
      space? '(' multiline_space?
      head:argument_list_item
      tail:(multiline_space? ',' multiline_space? argument_list_item)*
      multiline_space? ')' block_node:block? <ArgumentList>
      /
      space? '(' multiline_space? ')' block_node:block? <ArgumentList>
    end
    
    # No parentheses, so in order for it to definitely be an argument list we need:
    #
    #   1. EITHER 1 or more items
    #   2. OR no items, but a block
    #
    # Note that a block isn't included in option 1, as it would just bind to the last argument
    # anyway. If the block should bind to the method call, parentheses must be used.
    # 
    # The argument list may not start with + or -, so as not to confuse "x +1" with "x(+1)"
    rule unbracketed_argument_list
      space !('+' / '-')
      head:argument_list_item
      tail:(space? ',' multiline_space? argument_list_item)*
      block_node:block? <ArgumentList>
      /
      '' block_node:block <ArgumentList>
    end
    
    # A splat, a block pass, or a normal expression
    rule argument_list_item
      ('*' / '&')? multiline_space? expression <ArgumentListItem>
    end
    
    rule block
      braces_block / do_block
    end
    
    rule braces_block
      space? '{' block_argument_pattern multiline_space? expression_list '}' <Block>
    end
    
    rule do_block
      space 'do' block_argument_pattern multiline_space? expression_list 'end' <Block>
    end
    
    # A local or instance variable
    rule variable
      local_variable / instance_variable
    end
    
    # This is the same as local_variable except that it may also be a method call with an implicit
    # receiver, no parentheses and no arguments. This ambiguity has to be resolved at runtime.
    rule local_variable_or_method_call
      local_identifier '' <LocalVariableOrMethodCall>
    end
    
    # A local variable when we know it is definitely a variable (not a method)
    rule local_variable
      local_identifier '' <LocalVariable>
    end
    
    rule instance_variable
      '@' identifier <InstanceVariable>
    end
    
    rule keyword
      'class' / 'module' / 'def' /
      'do' / 'end' /
      'if' / 'else' / 'elsif' / 'while' /
      'begin' / 'rescue'
    end
    
    # All possible valid method names
    rule method_name
      (simple_method_name ('?' / '!' / '=')? / special_method_name) <MethodName>
    end
    
    # Method names which can be used in an implicit call. These are named which are *specifically
    # recognisable* as method calls (as opposed to local variables), hence a '?' or '!' at the end
    # is mandatory
    rule implicit_method_name
      simple_method_name ('?' / '!') <MethodName>
    end
    
    # Method names which can be used on the left of an assignment
    rule assignee_method_name
      '' simple_method_name <MethodName>
    end
    
    # Keywords are allowed
    rule simple_method_name
      [a-zA-Z_] [a-zA-Z0-9_]*
    end
    
    # Match in order of number of characters so we don't get conflict, e.g. if '<' is tested before
    # '<='
    rule special_method_name
      # Three characters
      '===' / '<=>' /
      '[]=' /
      
      # Two characters
      '==' / '!=' /
      '<=' / '>=' /
      '<<' / '>>' /
      '--' / '++' / '!!' /
      '[]' /
      
      # One character
      '<' / '>' /
      '+' / '-' /
      '*' / '/'
    end
    
    # A general identifier must not start with a number and must not conflict with a keyword
    rule identifier
      !keyword [a-zA-Z_] [a-zA-Z0-9_]* <Identifier>
    end
    
    # A local identifier must start with an underscore or lowercase letter, and must not conflict
    # with a keyword
    rule local_identifier
      !keyword [a-z_] [a-zA-Z0-9_]*
    end
    
    # A constant must start with a capital letter
    rule constant
      [A-Z] [a-zA-Z0-9_]* <Constant>
    end
    
    rule number
      [0-9]+ <Integer>
    end
    
    rule array
      '[' multiline_space?
      head:expression
      tail:(multiline_space? ',' multiline_space? expression)*
      multiline_space? ']' <Array>
      /
      '[' multiline_space? ']' <Array>
    end
    
    rule string
      string_without_interpolation /
      string_with_interpolation
    end
    
    rule string_without_interpolation
      "'" value:[^\']* "'" <StringWithoutInterpolation>
    end
    
    rule string_with_interpolation
      '"' value:[^\"]* '"' <StringWithInterpolation>
    end
    
    rule boolean
      'true' <True> / 'false' <False>
    end
    
    rule nil
      'nil' <Nil>
    end
    
    # A terminator signifies the end of a statement. It can be a newline or a semicolon, followed
    # by any amount of space
    rule terminator
      ("\n" / ";") multiline_space?
    end
    
    rule space
      [ \t]+
    end
    
    rule multiline_space
      [ \t\n\r]+
    end
  end
end

\end{lstlisting}
\begin{lstlisting}[title={\small\ttfamily\bfseries parser/nodes.rb},language=Ruby]
module Carat
  module Language
    module NodeHelper
      # The file is stored by the root node, so we delegate by to the parent by default and then
      # override this in Program
      def file_name
        parent.file_name
      end
      
      def line
        input.line_of(interval.first)
      end
      
      def column
        input.column_of(interval.first)
      end
      
      def location
        Carat::Location.new(file_name, line, column)
      end
      
      def error_location
        Carat::Location.new(file_name, line, column + 1)
      end
      
      def error(message)
        raise Carat::SyntaxError.new(input, message, error_location)
      end
    end
    
    class Treetop::Runtime::SyntaxNode
      include NodeHelper
    end
    
    class Program < Treetop::Runtime::SyntaxNode
      attr_accessor :file_name
      
      def to_ast
        expression_list.to_ast
      end
    end
  
    class ExpressionList < Treetop::Runtime::SyntaxNode
      # An array of nodes representing the expressions in the block
      def expressions
        [first] + rest.elements.map(&:expression)
      end
      
      def to_ast
        Carat::AST::ExpressionList.new(location, expressions.map(&:to_ast).compact)
      end
    end
    
    class EmptyExpressionList < Treetop::Runtime::SyntaxNode
      def to_ast
        nil
      end
    end
    
    class BracketedExpression < Treetop::Runtime::SyntaxNode
      def to_ast
        expression.to_ast
      end
    end
    
    class DefinitionNode < Treetop::Runtime::SyntaxNode
      def contents
        definition_body.expression_list.to_ast
      end
    end
    
    class ModuleDefinition < DefinitionNode
      def to_ast
        Carat::AST::ModuleDefinition.new(location, constant.text_value.to_sym, contents)
      end
    end
    
    class ClassDefinition < DefinitionNode
      def superclass_ast
        if superclass.empty?
          nil
        else
          superclass.primary.to_ast
        end
      end
      
      def to_ast
        Carat::AST::ClassDefinition.new(
          location, constant.text_value.to_sym,
          superclass_ast, contents
        )
      end
    end
    
    class MethodDefinition < DefinitionNode
      def receiver_ast
        !receiver.empty? && receiver.primary.to_ast || nil
      end
      
      def to_ast
        Carat::AST::MethodDefinition.new(
          location,
          receiver_ast, method_name.text_value.to_sym,
          method_argument_pattern.to_ast, contents
        )
      end
    end
    
    class IfExpression < Treetop::Runtime::SyntaxNode
      def false_expression_ast
        false_block.to_ast unless false_block.empty?
      end
      
      def true_expression_ast
        true_block.expression_list.to_ast
      end
    
      def to_ast
        Carat::AST::If.new(
          location, condition.to_ast,
          true_block.to_ast, false_expression_ast
        )
      end
    end
    
    class ElseExpression < Treetop::Runtime::SyntaxNode
      def to_ast
        expression_list.to_ast
      end
    end
    
    class WhileExpression < Treetop::Runtime::SyntaxNode
      def to_ast
        Carat::AST::While.new(location, condition.to_ast, contents.to_ast)
      end
    end
    
    class BeginExpression < Treetop::Runtime::SyntaxNode
      def rescue_ast
        self.rescue.to_ast unless self.rescue.empty?
      end
    
      def to_ast
        Carat::AST::Begin.new(location, contents.to_ast, rescue_ast)
      end
    end
    
    class RescueExpression < Treetop::Runtime::SyntaxNode
      def type_ast
        type.expression.to_ast unless type.empty?
      end
      
      def assignment_ast
        assignment.variable.to_ast unless assignment.empty?
      end
      
      def to_ast
        Carat::AST::Rescue.new(location, type_ast, assignment_ast, contents.to_ast)
      end
    end
    
    class ArgumentPattern < Treetop::Runtime::SyntaxNode
      def items
        @items ||= begin
          if contents.respond_to?(:head)
            items = [contents.head] + contents.tail.elements.map(&:item)
            items.compact.map(&:to_ast)
          else
            []
          end
        end
      end
      
      def splat_count
        items.find_all { |item| item.type == :splat }.length
      end
      
      def multiple_splats?
        splat_count > 1
      end
      
      def block_pass
        @block_pass ||= items.find { |item| item.type == :block_pass }
      end
      
      def block_pass_last?
        block_pass == items.last
      end
      
      def optional_part
        items.drop_while { |item| item.mandatory? }
      end
      
      def mandatory_before_optional?
        optional_part.find { |item| item.mandatory? }.nil?
      end
      
      def validate_items
        # There can only be one splat, otherwise there could be multiple ways to map arguments onto
        # the pattern
        if multiple_splats?
          error "only one splat allowed per method definition"
        end
        
        # This also implies there is only one block pass
        if block_pass && !block_pass_last?
          error "a block pass may only occur at the end of the argument list in a method definition"
        end
        
        unless mandatory_before_optional?
          error "all mandatory arguments must come before any optional ones"
        end
      end
      
      def to_ast
        if respond_to?(:contents)
          validate_items
          Carat::AST::ArgumentPattern.new(location, items)
        else
          Carat::AST::ArgumentPattern.new(location)
        end
      end
    end
    
    class ArgumentPatternItem < Treetop::Runtime::SyntaxNode
      def default_value_ast
        default.expression.to_ast if respond_to?(:default) && !default.empty?
      end
      
      def type
        case text_value.chars.first
          when '*'
            :splat
          when '&'
            :block_pass
          else
            :normal
        end
      end
      
      def to_ast
        Carat::AST::ArgumentPattern::Item.new(location, assignee.to_ast, type, default_value_ast)
      end
    end
    
    class Array < Treetop::Runtime::SyntaxNode
      def items
        if respond_to?(:head)
          [head] + tail.elements.map(&:expression)
        else
          []
        end
      end
      
      def to_ast
        Carat::AST::Array.new(location, items.map(&:to_ast))
      end
    end
    
    class String < Treetop::Runtime::SyntaxNode
      def to_ast
        Carat::AST::String.new(location, contents)
      end
    end
    
    class StringWithoutInterpolation < String
      def contents
        value.text_value
      end
    end
    
    class StringWithInterpolation < String
      def contents
        value.text_value.
          gsub('\n', "\n").
          gsub('\r', "\r").
          gsub('\t', "\t")
      end
    end
    
    class True < Treetop::Runtime::SyntaxNode
      def to_ast
        Carat::AST::True.new(location)
      end
    end
    
    class False < Treetop::Runtime::SyntaxNode
      def to_ast
        Carat::AST::False.new(location)
      end
    end
    
    class Nil < Treetop::Runtime::SyntaxNode
      def to_ast
        Carat::AST::Nil.new(location)
      end
    end
    
    class Integer < Treetop::Runtime::SyntaxNode
      def to_ast
        Carat::AST::Integer.new(location, text_value.to_i)
      end
    end
    
    class LocalVariable < Treetop::Runtime::SyntaxNode
      def to_ast
        Carat::AST::LocalVariable.new(location, text_value.to_sym)
      end
    end
    
    class LocalVariableOrMethodCall < Treetop::Runtime::SyntaxNode
      def to_ast
        Carat::AST::LocalVariableOrMethodCall.new(location, text_value.to_sym)
      end
    end
    
    class InstanceVariable < Treetop::Runtime::SyntaxNode
      def to_ast
        Carat::AST::InstanceVariable.new(location, identifier.text_value.to_sym)
      end
    end
    
    class MethodCallChain < Treetop::Runtime::SyntaxNode
      def chain
        [receiver] + tail.elements.map { |el| el.item }
      end
      
      # This basically resolves the associativity of a method call chain. During parsing, the chain
      # is matched based on right-bracketing, i.e. foo.[bar.[baz]]. However, the call to baz should
      # actually be the outermost node in the AST, and foo.bar is its receiver. So the bracketing
      # in the AST needs to be [[foo].bar].baz
      def reduce(chain)
        if chain.length == 1
          chain.first && chain.first.to_ast
        else
          call = chain.last
          receiver = reduce(chain[0..-2])
          
          if !call.respond_to?(:argument_list) || call.argument_list.empty?
            argument_list = Carat::AST::ArgumentList.new(location)
          else
            argument_list = call.argument_list.to_ast
          end
          
          Carat::AST::MethodCall.new(location, receiver, call.method_name.to_sym, argument_list)
        end
      end
      
      def to_ast
        reduce(chain)
      end
    end
    
    class ImplicitMethodCallChain < MethodCallChain
      def tail_elements
        if tail.empty?
          []
        else
          tail.elements.map { |el| el.item }
        end
      end
    
      def chain
        [nil, head] + tail_elements
      end
    end
    
    class AssigneeMethodCallChain < MethodCallChain
      def middle_elements
        if middle.empty?
          []
        else
          middle.elements.map { |el| el.method_call_segment.item }
        end
      end
      
      def chain
        [receiver] + middle_elements + [last.item]
      end
    end
    
    class ImplicitAssigneeMethodCallChain < AssigneeMethodCallChain
      def chain
        [nil, head] + middle_elements + [last.item]
      end
    end
    
    class ElementReference < Treetop::Runtime::SyntaxNode
      def method_name
        :[]
      end
      
      def items
        if respond_to?(:head)
          [head.to_ast] + tail.elements.map(&:argument_list_item).map(&:to_ast)
        else
          []
        end
      end
      
      def argument_list
        Carat::AST::ArgumentList.new(location, items)
      end
    end
    
    module MethodName
      def to_sym
        text_value.to_sym
      end
    end
    
    module Identifier
      def to_sym
        text_value.to_sym
      end
    end
    
    class UnaryMethodCall < Treetop::Runtime::SyntaxNode
      def to_ast
        Carat::AST::MethodCall.new(
          location, receiver.to_ast, (name.text_value * 2).to_sym,
          Carat::AST::ArgumentList.new(location)
        )
      end
    end
    
    class Assignment < Treetop::Runtime::SyntaxNode
      def receiver_ast
        @receiver_ast ||= receiver.to_ast
      end
      
      def value_ast
        value.to_ast
      end
      
      def to_ast
        Carat::AST::Assignment.new(location, receiver_ast, value_ast)
      end
    end
    
    module BinaryMethodHelper
      def method_call(left, name, right)
        Carat::AST::MethodCall.new(
          location, left.to_ast, name.text_value.to_sym,
          Carat::AST::ArgumentList.new(
            location, [
              Carat::AST::ArgumentList::Item.new(location, right.to_ast)
            ]
          )
        )
      end
    end
    
    class BinaryMethodCall < Treetop::Runtime::SyntaxNode
      include BinaryMethodHelper
      
      def to_ast
        method_call(left, name, right)
      end
    end
    
    class BinaryMethodAssignment < Assignment
      include BinaryMethodHelper
      
      def value_ast
        method_call(receiver_ast, name, value)
      end
    end
    
    class BinaryOperation < Treetop::Runtime::SyntaxNode
      OPERATIONS = {
        "&&" => Carat::AST::And,
        "||" => Carat::AST::Or
      }
      
      def to_ast
        OPERATIONS[name.text_value].new(location, left.to_ast, right.to_ast)
      end
    end
    
    class BinaryOperationAssignment < Assignment
      def value_ast
        BinaryOperation::OPERATIONS[name.text_value].new(location, receiver_ast, value.to_ast)
      end
    end
    
    class ArgumentList < Treetop::Runtime::SyntaxNode
      def items
        @items ||= begin
          items = []
          items += [head] + tail.elements.map(&:argument_list_item) if respond_to?(:head)
          items << block_node if respond_to?(:block_node) && !block_node.empty?
          items.map(&:to_ast)
        end
      end
      
      def block_pass
        items.find { |item| item.type == :block_pass }
      end
      
      def block
        items.last if items.last.type == :block
      end
      
      # Note that multiple splats are allowed, when *calling* a method, because there is no
      # ambiguity about how they will be expanded (the n items in the splat's expression will
      # become the next n items in the argument list). This is different to when defining a method
      # because a splat there means a certain variable will take an unbounded number of arguments
      # as a single array.
      def validate_items
        # Either a block may be passed, or a literal block may be created, but not both
        if block_pass && block
          error "cannot pass a block in the arguments and give a literal block at the same time"
        end
        
        # Block pass only valid at end of args. Note this also implies that multiple block passes
        # are invalid.
        if block_pass && block_pass != items.last
          error "a block pass must only occur at the end of the argument list"
        end
      end
      
      def to_ast
        validate_items
        Carat::AST::ArgumentList.new(location, items)
      end
    end
    
    class ArgumentListItem < Treetop::Runtime::SyntaxNode
      def type
        case text_value.chars.first
          when '*'
            :splat
          when '&'
            :block_pass
          else
            :normal
        end
      end
      
      def to_ast
        Carat::AST::ArgumentList::Item.new(location, expression.to_ast, type)
      end
    end
    
    class Block < Treetop::Runtime::SyntaxNode
      def to_ast
        Carat::AST::ArgumentList::Item.new(
          location,
          Carat::AST::Block.new(
            location, block_argument_pattern.to_ast,
            expression_list.to_ast
          ),
          :block
        )
      end
    end
    
    class Constant < Treetop::Runtime::SyntaxNode
      def to_ast
        Carat::AST::Constant.new(location, text_value.to_sym)
      end
    end
    
    class Nothing < Treetop::Runtime::SyntaxNode
      def to_ast
        nil
      end
    end
  end
end

\end{lstlisting}
\begin{lstlisting}[title={\small\ttfamily\bfseries parser/parser.rb},language=Ruby]
module Carat
  require "rubygems" rescue LoadError
  require "treetop"
  
  if ENV["DYNAMIC_PARSER"] == "true"
    Treetop.load(PARSER_PATH + "/comment")
    Treetop.load(PARSER_PATH + "/language")
  else
    require PARSER_PATH + "/comment"
    require PARSER_PATH + "/language"
  end
  
  require PARSER_PATH + "/nodes"

  class SyntaxError < CaratError
    attr_reader :input, :message, :location
    
    extend Forwardable
    def_delegators :location, :file_name, :line, :column
    
    def initialize(input, message, location)
      @input, @message, @location = input, message, location
    end
    
    # The input text on the offending line
    def line_contents
      input.split("\n")[line - 1]
    end
    
    # The line contents with an arrow underneath pointing at the column
    def diagram
      line_contents.to_s + "\n" + (" " * (column - 1)) + "^"
    end
    
    def full_message
      "#{location}: #{message}\n\n#{diagram}"
    end
  end
  
  class ParseError < SyntaxError; end
  
  # Adapts the parser to store the code and the file name. This means we can break up the process of
  # parsing a bit more easily.
  class LanguageParser < Treetop::Runtime::CompiledParser
    attr_reader :input, :file_name
    
    def initialize(input, file_name)
      super()
      @input, @file_name = input, file_name
    end
    
    # Parses the code converts it to an AST, raising syntax errors along the way if necessary
    def ast
      @ast ||= begin
        parse_tree.file_name = file_name
        parse_tree.to_ast
      end
    end
    
    def parse_tree
      @parse_tree ||= begin
        tree = parse(input_without_comments)
        if tree.nil?
          raise Carat::ParseError.new(
            input, expected_message,
            Carat::Location.new(file_name, failure_line, failure_column)
          )
        end
        tree
      end
    end
    
    def input_without_comments
      parser = CommentParser.new
      parse_tree = parser.parse(@input)
      
      unless parse_tree
        puts "Comment parser failed"
        puts parser.failure_reason
        exit 1
      end
      
      parse_tree.strip
    end
    
    def expected_message
      tf = terminal_failures
      message = "Expected "
      message << "one of " if tf.length > 1
      message << tf.map(&:expected_string).uniq.join(', ')
    end
  end
end

\end{lstlisting}
\begin{lstlisting}[title={\small\ttfamily\bfseries repl.rb},language=Ruby]
require 'readline'

module Carat
  class REPL
    def initialize
      @runtime    = Carat::Runtime.new
      @scope      = @runtime.main_scope
      @expression = ""
    end
    
    def run
      puts "Welcome to Carat."
      loop { readline }
    end
    
    def readline
      line = Readline.readline(prompt)
      exit(0) if line == "exit"
      
      @expression << line + "\n"
      
      begin
        ast = Carat.parse(@expression)
        result = @runtime.execute(ast, @scope)
        inspected_result = @runtime.with_stack do
          result.call(:inspect, &@runtime.identity_continuation)
        end
        
        puts "=> " + inspected_result.to_s
        @expression = ""
      rescue SyntaxError
        # It's assumed that syntax errors mean the expression is incomplete, so we just rescue
        # them and carry on
      end
    rescue Interrupt
      if @expression.empty?
        # Exit silently
        puts
        exit(0)
      else
        # Reset the current expression, but don't exit the REPL
        @expression = ""
        puts
      end
    end
    
    def prompt
      if @expression.empty?
        ">> "
      else
        "?> "
      end
    end
  end
end

\end{lstlisting}
\begin{lstlisting}[title={\small\ttfamily\bfseries runtime/call.rb},language=Ruby]
class Carat::Runtime
  class Arguments
    attr_accessor :values, :block
    
    def initialize(values = [], block = nil)
      @values, @block = values, block
    end
    
    # Assumes the last item in the array is a block or NilClassInstance representing no block
    def self.from_a(arguments)
      block = arguments.pop
      block = nil if block.is_a?(Carat::Data::NilClassInstance)
      new(arguments, block)
    end
    
    def to_a
      (@values + [block]).compact
    end
  end
  
  class AbstractCall
    # The runtime in which the call is happening
    attr_reader :runtime
    
    # The object representing whatever we are calling (method, lambda, primitive method, etc)
    attr_reader :callable
    
    # The scope in which the Call was created, used for evaluating arguments
    attr_reader :caller_scope
    
    # The argument list can come in various forms, see eval_arguments
    attr_reader :argument_list
    
    # The continuation of this call - i.e. the computation to be done afterwards
    attr_reader :continuation
    
    def initialize(runtime, callable, argument_list, continuation)
      @caller_scope = runtime.current_scope
      
      @runtime,       @callable     = runtime,       callable
      @argument_list, @continuation = argument_list, continuation
    end
    
    def send
      raise NotImplementedError
    end
    
    private
    
      # This method returns an Arguments object. Before "evaluation" the arguments may be one of
      # three things:
      # 
      #   1. An Arguments object; just yield immediately
      #   2. An Array; pass to a new Arguments object and yield
      #   3. An ArgumentList AST node; evaluate it
      def eval_arguments(&continuation)
        case argument_list
          when Arguments
            yield argument_list
          when Array
            yield Arguments.new(argument_list)
          else
            argument_list.eval_in_scope(caller_scope, &continuation)
        end
      end
  end
  
  class PrimitiveCall < AbstractCall
    def send
      eval_arguments do |arguments|
        callable.call(*arguments.to_a, &return_continuation)
      end
    end
    
    def return_continuation
      @return_continuation ||= lambda do |result|
        unless result.is_a?(Carat::Data::ObjectInstance)
          raise Carat::CaratError, "primitive '#{name}' did not return an ObjectInstance: #{result.inspect}"
        end
        
        continuation.call(result)
      end
    end
  end
  
  class Call < AbstractCall
    # Scope used for the evaluation of the call
    attr_reader :scope
    
    # Location that the call was made at
    attr_reader :location
    
    extend Forwardable
    def_delegators :callable, :argument_pattern, :contents
    
    def initialize(runtime, callable, argument_list, continuation, scope, location)
      super(runtime, callable, argument_list, continuation)
      @scope, @location = scope, location
    end
    
    def send
      runtime.stack << frame
      apply_arguments do
        if contents.nil?
          return_continuation.call(runtime.nil)
        else
          lambda { contents.eval(&return_continuation) }
        end
      end
    end
    
    def return_continuation
      @return_continuation ||= lambda do |result|
        runtime.stack.pop
        continuation.call(result)
      end
    end
    
    def to_s
      callable.to_s
    end
    
    def inspect
      "Call[#{callable}, #{location}]"
    end
    
    private
    
      def frame
        @frame ||= Frame.new(scope, self)
      end
      
      def apply_arguments(&continuation)
        eval_arguments do |arguments|
          scope.block = arguments.block unless arguments.block.nil?
          argument_pattern.assign(arguments.values.clone, &continuation)
        end
      end
  end
  
  class MainMethodCall
    attr_reader :location
    
    def initialize(location)
      @location = location
    end
    
    def to_s
      "main"
    end
    
    def inspect
      "Call[main]"
    end
  end
end

\end{lstlisting}
\begin{lstlisting}[title={\small\ttfamily\bfseries runtime/kernel\_loader.rb},language=Ruby]
class Carat::Runtime
  class KernelLoader
    attr_reader :runtime
  
    extend Forwardable
    def_delegators :runtime, :constants
    
    include Carat::Data
    
    LOAD_ORDER = [:kernel, :module, :class, :object, :comparable, :fixnum, :array, :string,
                  :nil_class, :true_class, :false_class, :lambda, :exception]
    
    def initialize(runtime)
      @runtime = runtime
    end
    
    def run
      # Create SingletonClass and Object. The super pointers of SingletonClass, SingletonClass',
      # Object and Object' will be nil. The klass pointers of SingletonClass' and Object' will also
      # be nil.
      @singleton_class = constants[:SingletonClass] = SingletonClassClass.new(runtime, nil, nil)
      @object          = constants[:Object]         = ObjectClass.new(runtime, nil, nil)
      
      # Set the klass pointers of SingletonClass' and Object'
      # The class of any singleton class is SingletonClass
      @singleton_class.singleton_class.klass = @singleton_class
      @object.singleton_class.klass          = @singleton_class
      
      # Create Module and Class. The super and klass pointers can be inferred correctly at this 
      # stage based on the superclass given.
      @module = constants[:Module] = ModuleClass.new(runtime, nil, @object)
      @class  = constants[:Class]  = ClassClass.new(runtime, nil, @module)
      
      # Special case: The super of Object's singleton class is Class
      @object.singleton_class.super = @class
      
      # Now position SingletonClass as a subclass of Class
      @singleton_class.super = @class
      @singleton_class.singleton_class.super = @class.singleton_class
      
      constants[:Kernel] = ModuleInstance.new(runtime, @module, :Kernel)
      
      create_classes(:Primitive, :Fixnum, :Array, :String, :Lambda, :Method,
                     :NilClass, :TrueClass, :FalseClass, :SingletonClass)
      
      LOAD_ORDER.each do |file|
        runtime.execute(Marshal.load(File.read(Carat::KERNEL_PATH + "/#{file}.marshal")))
      end
    end
    
    def create_classes(*names)
      names.each do |name|
        constants[name] = self.class.const_get("#{name}Class").new(runtime, @class, @object)
      end
    end
  end
end

\end{lstlisting}
\begin{lstlisting}[title={\small\ttfamily\bfseries runtime/runtime.rb},language=Ruby]
module Carat
  class Runtime
    require RUNTIME_PATH + "/kernel_loader"
    require RUNTIME_PATH + "/scope"
    require RUNTIME_PATH + "/call"
    require RUNTIME_PATH + "/stack"
    
    attr_reader :stack, :constants, :loaded_files
    
    extend Forwardable
    def_delegators :stack, :current_scope, :current_call, :current_failure_continuation
    
    def initialize
      # When a new file is run it needs a new stack. But we need to be able to return to the 
      # previous file once that file has been run. So for that we need a stack of stacks.
      @stack_of_stacks = []
      
      # Constants are defined globally
      @constants = {}
      
      # Keep track of which additional files have been loaded
      @loaded_files = []
      
      # Load core classes
      KernelLoader.new(self).run
    end
    
    def stack
      @stack_of_stacks.last
    end
    
    def current_location
      current_call && current_call.location
    end
    
    def current_object
      current_scope[:self]
    end
    
    # Returns a list of calls from the stack. (Note that not all stack frames have a call associated
    # with them.)
    def call_stack
      stack.to_a.map(&:call).compact
    end
    
    def false
      constants[:FalseClass].instance
    end
    
    def true
      constants[:TrueClass].instance
    end
    
    def nil
      constants[:NilClass].instance
    end
    
    # This is similar to a 'foldl' or 'inject' function, but written for this specific context
    # where we are using continuation passing style
    def fold(current_answer, operation, items, start = 0, &continuation)
      if start == items.length
        # ** Base Case ** #
        # There are no items to process because we have got to the end of the array, so yield the
        # current answer to the continuation, taking us out of the fold operation
        yield current_answer
      else
        # ** Inductive Case ** #
        # Pass the first item in items[start...items.length], along with the current answer, to the
        # operation. The operation should combine them in some way to form the next answer, before
        # yielding to its continuation, which will then fold items[(start+1)...items.length].
        operation.call(items[start], current_answer) do |next_answer|
          lambda do
            fold(next_answer, operation, items, start + 1, &continuation)
          end
        end
      end
    end
    
    # This is an 'each' function written in continuation passing style
    def each(operation, items, final_answer = true, start = 0, &continuation)
      if start == items.length
        yield final_answer
      else
        operation.call(items[start]) do
          lambda do
            each(operation, items, final_answer, start + 1, &continuation)
          end
        end
      end
    end
    
    # Raises an exception in the object language
    def raise(exception_name, message, location = current_location)
      constants[exception_name].call(:new, [constants[:String].new(message)]) do |exception|
        exception.generate_backtrace(location)
        stack.unwind_to(:failure_continuation)
        current_failure_continuation.call(exception)
      end
    end
    
    def default_failure_continuation
      lambda do |exception|
        exception.call(:to_s) do |exception_string|
          puts "#{exception.real_klass.name}: #{exception_string}"
          puts exception.backtrace.map { |line| "  " + line }.join("\n")
          exit 1
        end
      end
    end
    
    def identity_continuation
      lambda { |x| x }
    end
    
    def main_scope
      Scope.new(constants[:Object].new)
    end
    
    def call_main_method(contents, scope = nil)
      call  = MainMethodCall.new(contents.location)
      frame = Frame.new(scope || main_scope, call, default_failure_continuation)
      
      contents.runtime = self
      contents.eval_in_frame(frame, &identity_continuation)
    end
    
    # Normally, in Continuation Passing Style, a stack of continuations is built up right until the
    # end of the program when they all collapse in to provide the result. In languages without tail
    # call optimisation (such as Ruby 1.8), this quickly leads to an enormous stack, and it's not
    # hard to create programs which cause the interpreter to run out of stack space.
    # 
    # Therefore, instead of waiting until right at the end of the program to return the answer,
    # we can at any point return a "partial answer" which is just a lambda which can be called to
    # continue the execution of the program. This collapses the call stack right back down, so
    # solves the problem of tail call recursion. This is what the while loop is doing. This
    # technique is called "trampolining".
    def with_stack(&result)
      @stack_of_stacks << Stack.new
      while result.is_a?(Proc)
        result = result.call
      end
      @stack_of_stacks.pop
      result
    end
    
    # This is the starting point for executing an AST node. It places the node as the contents of
    # a special "main" method and then evaluated that method.
    def execute(root, scope = nil)
      with_stack { call_main_method(root, scope) }
    end
    
    # Parse some code and then execute its AST
    def run(input, file_name = nil)
      execute(Carat.parse(input, file_name))
    rescue StandardError => e
      handle_error(e)
    end
    
    # Read the contents of a file and run it
    def run_file(name)
      run(File.read(name), name)
    end
    
    private
      
      def handle_error(exception)
        case exception
          when SyntaxError
            puts exception.full_message
          else
            puts "Error: #{exception.message}"
            puts exception.backtrace[0..40].join("\n")
            puts "[Backtrace truncated]" if exception.backtrace.length > 40
            puts
            puts "Call Stack"
            puts "=========="
            puts
            puts call_stack.reverse.map(&:inspect).join("\n")
        end
        exit 1
      end
  end
end

\end{lstlisting}
\begin{lstlisting}[title={\small\ttfamily\bfseries runtime/scope.rb},language=Ruby]
class Carat::Runtime
  class Scope
    attr_reader :symbols, :parent
    attr_writer :block
    
    def initialize(self_object, parent = nil)
      raise ArgumentError if self_object.nil?
      
      @symbols = { :self => self_object }
      @parent  = parent
    end
    
    # Get a symbol from this scope or parent scope. May return nil.
    def [](symbol)
      @symbols[symbol] || @parent && @parent[symbol]
    end
    
    # Assign a value to a symbol
    def []=(symbol, value)
      if @symbols.has_key?(symbol)     # If it exists here, assign it here
        @symbols[symbol] = value
      elsif @parent && @parent[symbol] # If it exists in a parent, assign it there
        @parent[symbol] = value
      else                             # Otherwise initialise a fresh symbol here
        @symbols[symbol] = value
      end
    end
    
    # Assign a hash of multiple symbol and value pairs
    def merge!(items)
      items.each do |symbol, value|
        self[symbol] = value
      end
    end
    
    # Get the current block in this scope - this is inherited from parent scopes if there is no
    # block set in this scope
    def block
      @block || @parent && @parent.block
    end
    
    # Create a new scope using this one as the parent
    def extend(self_object = nil)
      self_object ||= self[:self]
      Scope.new(self_object, self)
    end
    
    def inspect
      @symbols.inspect
    end
  end
end

\end{lstlisting}
\begin{lstlisting}[title={\small\ttfamily\bfseries runtime/stack.rb},language=Ruby]
class Carat::Runtime
  # A stack frame can contain a scope, a call, and a failure continuation. All are optional, but it
  # is expected that a frame will contain at least one of these (otherwise it is pretty useless).
  class Frame
    attr_reader :scope, :call, :failure_continuation
    
    def initialize(scope = nil, call = nil, failure_continuation = nil)
      @scope, @call, @failure_continuation = scope, call, failure_continuation
    end
  end
  
  # The stack contains a number of stack frames. It has a current scope, current call and current
  # failure continuation, which is taken from the frame nearest the top of the stack which has
  # the desired attribute.
  class Stack
    def initialize
      @items = []
    end
    
    def <<(item)
      @items << item
      invalidate_cache
      self
    end
    
    def current_scope
      @current_scope ||= find_last(:scope)
    end
    
    def current_call
      @current_call ||= find_last(:call)
    end
    
    def current_failure_continuation
      @current_failure_continuation ||= find_last(:failure_continuation)
    end
    
    def pop
      item = @items.pop
      invalidate_cache
      item
    end
    
    # Pop frames from the stack until we get to the first frame with the given attribute
    def unwind_to(attribute)
      frame = @items.last
      while frame.send(attribute).nil?
        @items.pop
        frame = @items.last
      end
      invalidate_cache
      frame
    end
    
    def to_a
      @items.clone
    end
    
    private
    
      def invalidate_cache
        @current_scope = nil
        @current_call = nil
        @current_failure_continuation = nil
      end
      
      # Find the frame nearest to the top of the stack where the given attribute is non-nil
      def find_last(attribute)
        i = @items.length - 1
        i -= 1 while i >= 0 && @items[i].send(attribute).nil?
        @items[i].send(attribute) if i >= 0
      end
  end
end

\end{lstlisting}

\end{document}

